\documentclass[11pt,usenames,dvipsnames,reqno]{amsart} 
%\usepackage[margin=1.34in]{geometry}
\usepackage{xypic}
\usepackage{amsmath}
\usepackage{amssymb}
\usepackage{amsthm}
%\usepackage[notref,notcite]{showkeys}
\usepackage[colorlinks=true,linkcolor=NavyBlue,citecolor=NavyBlue,anchorcolor=NavyBlue,urlcolor=NavyBlue]{hyperref}
\usepackage{graphicx}
\usepackage{enumitem}
\usepackage{tikz}

%Shortcuts
\newcommand{\cc}{\mathbb{C}}
\newcommand{\zz}{\mathbb{Z}}
\newcommand{\rr}{\mathbb{R}}
\renewcommand{\sl}{\mathrm{sl}}
\newcommand{\tb}{\mathrm{tb}}
\newcommand{\lk}{\mathrm{lk}}
\newcommand{\st}{{\mathrm{st}}}
\newcommand{\id}{\operatorname{id}}
\newcommand{\pr}{\operatorname{pr}}
\newcommand{\Symp}{\operatorname{Symp}}
\newcommand{\Mod}{\operatorname{Mod}}
\newcommand{\Diff}{\operatorname{Diff}}
\newcommand{\km}{{\normalfont{(KM)}}}
\newcommand{\vol}{\operatorname{vol}}


%Environments
\newtheoremstyle{thm}{15 pt}{10 pt}{\itshape}{}{\bfseries}{.}{.5em}{}
\newtheorem{thm-main}{Theorem}
\newtheorem{theorem}{Theorem}
\numberwithin{theorem}{section}
\newtheorem{corollary}[theorem]{Corollary}
\newtheorem{question}[theorem]{Question}
\newtheorem{lemma}[theorem]{Lemma}
\newtheorem{conjecture}[theorem]{Conjecture}
\newtheorem{proposition}[theorem]{Proposition}
\newtheorem*{thm*}{Theorem}
\newtheorem*{cor*}{Corollary}

\newtheoremstyle{ex}{10 pt}{10 pt}{\normalfont}{}{\bfseries}{.}{.5em}{}
\theoremstyle{ex}
\newtheorem{example}[theorem]{Example}
\newtheorem{definition}[theorem]{Definition}

\newtheoremstyle{rem}{10 pt}{10 pt}{\normalfont}{}{\it}{.}{.5em}{}
\theoremstyle{rem}
\newtheorem{remark}[theorem]{Remark}

\usepackage{xcolor}
\def\kh#1{\textcolor{Blue}{#1}}
\begin{document}

%%%%%%%%%%%%%%%%%%%%%%%%
%                     End of preamble                       %
%%%%%%%%%%%%%%%%%%%%%%%%


\title{Integer characterizing slopes and unknotting numbers}

\author[G. Agostini, S. Chen, K. Hayden, C. Serio, C. Wang, A. Wu, and K. Wu]{Gabriel Agostini, Sophia Chen, Kyle Hayden, Christian Serio, Cecilia Wang, Anton Wu, and Kexin Wu}



\begin{abstract} 
\kh{[Write one!]}
\end{abstract}


\maketitle

\section{Introduction}\label{sec:intro}

\kh{[Introduction with some motivation and introduction of terms like ``characterizing slopes''. See the papers by Yi and Zhang and by McCoy and by Lackenby on characterizing slopes for ideas to steal. Mention Piccirillo's results for unknotting number one and using this to show that the Conway knot isn't slice. Mention that McCoy's work \cite{mccoy:hyperbolic}, which shows that a hyperbolic knot has only finitely many non-characterizing slopes $p/q$ with $|q| \geq 3$. In a sense, this implies that ``most'' slopes $p/q$ are characterizing for any given hyperbolic knot $K$: The  probability that a randomly chosen slope $p/q$ is characterizing approaches 1 as $|p|+|q|\to \infty$.]}



\begin{theorem}\label{thm:unknotting-one} If a knot $K$ has unknotting number one and is not a twisted Whitehead double, then $K$ has at most finitely many integer characterizing slopes.
\end{theorem}

\kh{[Mention the following findings for knots with low crossing number. Also mention that it shows that having at most finitely many integer characterizing slopes is not special to knots with unknotting number one.]}


\begin{theorem}\label{thm:low-crossing} For knots $K$ with crossing number $c(K) \leq 10$:
\begin{enumerate}[label=\normalfont \bf (\alph*)]
\item If $K$ has unknotting number $u(K)=1$ and $K$ is not a twist knot, then $K$ has at most one integer characterizing slope.
\item If $K$ is one of the knots $8_4$, $8_6$, $8_{10}$, or $8_{12}$, then $K$ has $u(K)>1$ and has no integer characterizing slopes.
\end{enumerate}
\end{theorem}

\kh{[Can we push this to $c(K)\leq 12$? Can we expand on the list of knots with $u(K)>1$?]} 
\kh{[Mention that these results affirmatively answer a question of Baker and Motegi \cite[Question~1.7]{baker-motegi}, who asked if there exist knots with fewer than 8 crossings that have infinitely many noncharacterizing slopes.]}


\begin{theorem}\label{thm:twist} For the twist knot $8_1$, all nonzero integer slopes are non-characterizing.
\end{theorem}

\kh{[Remark: This shows that the assumption in Theorem 1.2 that $K$ is not a twisted Whitehead double is not necessary.]}






\subsection{Non-characterizing slopes and dual knots} Let $Y$ be a 3-manifold given by $p/q$-surgery on a knot $K\subset S^3$. Then $Y$ is obtained by thickening $K$ to a solid torus $N(K)$, removing the interior of $N(K)$ from $S^3$, and gluing a solid torus back into $S^3\setminus\mathrm{int}(N(K))$ in a different manner. Note that $K$ is the core of the solid torus $N(K)$, i.e., $K$ is identified with $S^1\times\{0\}$ under the diffeomorphism between $N(K)$ and $S^1\times D^2$. The core of the new solid torus that is glued into $S^3\setminus\mathrm{int}(N(K))$ specifies a knot $\gamma$ in $Y$, which we call the \textit{surgery dual} of $K$. In a surgery diagram for $Y$, $\gamma$ can be represented by a meridian to $K$. If $K'\subset S^3$ is another knot which yields the same $p/q$-surgery as $K$, then one can ask how the surgery dual $\gamma'$ of $K'$ in $Y$ is related to $\gamma$. 

\begin{conjecture}[Baker] If $K$ and $K'$ are non-isotopic knots in $S^3$ which yield the same $3$-manifold $Y$ under $p/q$-surgery, then their surgery duals $\gamma$ and $\gamma'$ in $Y$ are not homotopic.
\end{conjecture}

\subsection{Code and data} \kh{[Summarize how computer calculations factor into the proof of Theorem~\ref{thm:low-crossing} and the results on Baker's conjecture. Ultimately, we'll make the files publicly available, and this subsection can describe where to find those files.]} \kh{[Also, we can probably get Gabriel's python script to work for any link that fits into Piccirillo's construction. That would be useful.]}



\section{Knots with unknotting number one}\label{sec:unknotting-one}

\subsection{Banded Hopf link diagrams} \kh{[Describe these and how they produce diagrams that fit into Baker and Motegi's construction. Give examples for a couple knots with unknotting number one, plus an example for a knot with unknotting number two.]}

\subsection{Piccirillo's construction}

\subsection{Twist families of knots}

\kh{[Define what it means to produce a twist family of knots $K_n$ from a two-component link $K\cup C$ where $C$ is an unknot that is linked with $K$. A picture would help.]}

\kh{[Recall Theorem~3.2 from KMS.]}

\subsection{Proof of Theorem~\ref{thm:unknotting-one}}

\section{Knots with low crossing number}


\subsection{Hyperbolic Dehn surgery} \kh{[Recall and discuss the relevant theorems used in our approach.]}

\subsection{Proof of Theorem~\ref{thm:low-crossing}}

\kh{[Desribe the two-part algorithm we use to rule out integer characterizing slopes, beginning with finding $N \in \zz_{>0}$ such that $\vol(S^3 \setminus K'_n)> \vol(S^3 \setminus K)$ for all $|n|>N$ and then directly examining $K'_n$ for $|n| \leq N$.]}

\section{Additional results}

\kh{[Any additional findings, including HFK or Khovanov homology findings, or enhanced sliceness obstructions using Piccirillo's technique, can go here.]}


\bibliographystyle{alpha}
\bibliography{biblio}

\end{document}
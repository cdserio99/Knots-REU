\documentclass[11pt,usenames,dvipsnames,reqno]{amsart} 
%\usepackage[margin=1.34in]{geometry}
\usepackage{xypic}
\usepackage{amsmath}
\usepackage{amssymb}
\usepackage{amsthm}
%\usepackage[notref,notcite]{showkeys}
\usepackage[colorlinks=true,linkcolor=NavyBlue,citecolor=NavyBlue,anchorcolor=NavyBlue,urlcolor=NavyBlue]{hyperref}
\usepackage{graphicx}
\usepackage{enumitem}
\usepackage{tikz}

%Shortcuts
\newcommand{\cc}{\mathbb{C}}
\newcommand{\zz}{\mathbb{Z}}
\newcommand{\rr}{\mathbb{R}}
\renewcommand{\sl}{\mathrm{sl}}
\newcommand{\tb}{\mathrm{tb}}
\newcommand{\lk}{\mathrm{lk}}
\newcommand{\st}{{\mathrm{st}}}
\newcommand{\id}{\operatorname{id}}
\newcommand{\pr}{\operatorname{pr}}
\newcommand{\Symp}{\operatorname{Symp}}
\newcommand{\Mod}{\operatorname{Mod}}
\newcommand{\Diff}{\operatorname{Diff}}
\newcommand{\km}{{\normalfont{(KM)}}}
\newcommand{\vol}{\operatorname{vol}}


%Environments
\newtheoremstyle{thm}{15 pt}{10 pt}{\itshape}{}{\bfseries}{.}{.5em}{}
\newtheorem{thm-main}{Theorem}
\newtheorem{theorem}{Theorem}
\numberwithin{theorem}{section}
\newtheorem{corollary}[theorem]{Corollary}
\newtheorem{question}[theorem]{Question}
\newtheorem{lemma}[theorem]{Lemma}
\newtheorem{conjecture}[theorem]{Conjecture}
\newtheorem{proposition}[theorem]{Proposition}
\newtheorem*{thm*}{Theorem}
\newtheorem*{cor*}{Corollary}

\newtheoremstyle{ex}{10 pt}{10 pt}{\normalfont}{}{\bfseries}{.}{.5em}{}
\theoremstyle{ex}
\newtheorem{example}[theorem]{Example}
\newtheorem{definition}[theorem]{Definition}

\newtheoremstyle{rem}{10 pt}{10 pt}{\normalfont}{}{\it}{.}{.5em}{}
\theoremstyle{rem}
\newtheorem{remark}[theorem]{Remark}

\usepackage{xcolor}
\def\kh#1{\textcolor{Blue}{#1}}
\begin{document}

%%%%%%%%%%%%%%%%%%%%%%%%
%                     End of preamble                       %
%%%%%%%%%%%%%%%%%%%%%%%%


\title{Integer characterizing slopes and unknotting numbers}

\author[G. Agostini, S. Chen, K. Hayden, C. Serio, C. Wang, A. Wu, and K. Wu]{Gabriel Agostini, Sophia Chen, Kyle Hayden, Christian Serio, Cecilia Wang, Anton Wu, and Kexin Wu}



\begin{abstract} 
\kh{[Write one!]}
\end{abstract}


\maketitle

\section{Introduction}\label{sec:intro}

\kh{[Introduction with some motivation and introduction of terms like ``characterizing slopes.'' See the papers by Yi and Zhang and by McCoy and by Lackenby on characterizing slopes for ideas to steal. Mention Piccirillo's results for unknotting number one and using this to show that the Conway knot isn't slice. Mention that McCoy's work \cite{mccoy:hyperbolic}, which shows that a hyperbolic knot has only finitely many non-characterizing slopes $p/q$ with $|q| \geq 3$. In a sense, this implies that ``most'' slopes $p/q$ are characterizing for any given hyperbolic knot $K$: The  probability that a randomly chosen slope $p/q$ is characterizing approaches 1 as $|p|+|q|\to \infty$.]}

\begin{theorem}\label{thm:unknotting-one} If a knot $K$ has unknotting number $u(K)=1$ and is not a twisted Whitehead double, then $K$ has at most finitely many integer characterizing slopes.
\end{theorem}

We worked our way towards Theorem ~\ref{thm:unknotting-one} by manually finding knots $K_{n}'$ with the same $n$-surgery as a given knot $K$ with $u(K)=1$. Beginning with this process, we found the following result for knots with low crossing number:

\begin{theorem}\label{thm:low-crossing} For knots $K$ with crossing number $c(K) \leq 10$:
	\begin{enumerate}[label=\normalfont \bf (\alph*)]
		\item If $K$ has unknotting number $u(K)=1$ and $K$ is not a twist knot, then $K$ has at most one integer characterizing slope, namely $\pm 2$.
		\item If $K$ is one of the knots $8_4$, $8_6$, $8_{10}$, or $8_{12}$, then $K$ has $u(K)>1$ and has no integer characterizing slopes.
		\item If $K$ is the twist knot $8_1$, then $K$ has at most one integer characterizing slope, namely $0$.
	\end{enumerate}
\end{theorem}

This theorem specifies the number of integer characterizing slopes for several knots with $c(K) \leq 10$. It is possible that the bound on $c(K)$ can be increased, especially for knots with unknotting number $u(K) = 1$. Similarly, the list in part (b) can probably be expanded, encompassing every knot that fits Piccirilo's construction. Part (c) of this theorem also suggests that the assumption in Theorem ~\ref{thm:unknotting-one} that $K$ is not a twisted Whitehead double might is not a necessary condition. Moreover, it was proven \kh{by Ozsv\'{a}th and Szab\'{o} 2006, maybe cite}  that every slope is characterizing for the trefoil and the figure-eight knot. Part (c) shows that this fact does not generalize to twist knots with higher crossing numbers.

Theorem~\ref{thm:low-crossing} affirmatively answers a question posed by Baker and Motegi \cite[Question~1.7]{baker-motegi}:

\begin{question}
	Are there any knots of crossing number less than 8 that have infinitely many noncharacterizing slopes? 
\end{question}

\begin{corollary}
	All knots of crossing number less than 8 with unknotting number one which are not twist knots have infinitely many non-characterizing slopes.
\end{corollary}

The proof of this theorem relies in part on a computer program to calculate volumes of finitely many surgeries on each knot. We use the software SnapPy, which can triangulate hyperbolic manifolds. The script used is included on \kh{maybe put on Appendix?}, and the files produced for all knots encompassed can be found on \kh{Kyle's website}.

\subsection{Surgery duals} Let $Y$ be a 3-manifold given by $p/q$-surgery on a knot $K\subset S^3$. Then $Y$ is obtained by thickening $K$ to a solid torus $N(K)$, removing the interior of $N(K)$ from $S^3$, and gluing a solid torus back into $S^3\setminus\mathrm{int}(N(K))$ in a different manner. Note that $K$ is the core of the solid torus $N(K)$, i.e., $K$ is identified with $S^1\times\{0\}$ under the diffeomorphism between $N(K)$ and $S^1\times D^2$. The core of the new solid torus that is glued into $S^3\setminus\mathrm{int}(N(K))$ specifies a knot $\gamma$ in $Y$, which we call the \textit{surgery dual} of $K$. In a surgery diagram for $Y$, $\gamma$ can be represented by a meridian to $K$. If $K'\subset S^3$ is another knot which yields the same $p/q$-surgery as $K$, then one can ask how the surgery dual $\gamma'$ of $K'$ in $Y$ is related to $\gamma$. 

\begin{conjecture}[Baker] If $K$ and $K'$ are non-isotopic knots in $S^3$ which yield the same $3$-manifold $Y$ under $p/q$-surgery, then their surgery duals $\gamma$ and $\gamma'$ in $Y$ are not homotopic.
\end{conjecture}

We showed that Baker's conjecture holds for all knots $K$ with unknotting number $u(K) = 1$ and crossing number $c(K) \leq 10$ in the case of zero surgery. The files corresponding to this verification can be found on \kh{Kyle's website}.


\section{Knots with unknotting number one}\label{sec:unknotting-one}


\subsection{Banded Hopf link diagrams} A knot K with unknotting number one can be represented by a diagram that consists of a Hopf link with its two components connected by a band. We have the following theorem:

\begin {theorem} If $K$ has unknotting number one, then it is obtained from the Hopf link by a single band move.
\end{theorem}

To find the banded Hopf link diagram for a knot $K$ with unknotting number one, we first find the unkotting crossing of $K$ and alternate it. We then draw a meridian adjacent to it and slide $K$ over the meridian. After simplification, we obtain the banded Hopf link diagram for $K$. 

We now present a characterization of twisted Whitehead doubles in terms of their band presentations:


\begin{theorem}\label{thm:twd} A knot $K$ is a twisted Whitehead double if and only if it has a band presentation in which the band does not cross either component of the Hopf link.

\end{theorem}

\begin{theorem}\label{thm:twd} A knot $K$ is a twisted Whitehead double if and only if $K$ has a band presentation in which the band does not cross either component of the Hopf link.

\end{theorem}
\begin{proof}
	\kh{[Is a proof necessary here? Can we find a resource?]}
\end{proof}

\subsection{Piccirillo's construction} We have the following theorem due to Piccirillo:

\begin{theorem}[Piccirillo, 2018]
	Let $L$ be a three-component link consisting of disjoint components $R$, $B$, and $G$, giving a surgery diagram such that
	\begin{enumerate}[label=\normalfont (\alph*)]
		\item $R$ is a 0-framed unknot,
		\item $B$ and $G$ have integral framings,
		\item If we remove $G$ (resp. $B$), then $R$ is isotopic to a meridian to $B$ (resp. $G$).
	\end{enumerate}
	Let $Y$ be the $3$-manifold given by surgery on $L$. Then for any $n\in\mathbb{Z}$, there are knots $K,K'\subset S^3$ such that $Y\cong S_n^3(K)\cong S_n^3(K')$.

\end{theorem}

If $K$ is a knot with $u(K)=1$, then for any integer $n$, this construction yields another knot $K_n'$ (not necessarily distinct from $K$) with the same $n$-surgery as $K$. We begin with a banded Hopf link diagram for $K$. We take a Hopf link with components $R$ and $B$, both with framings 0, and we handle slide $B$ over $R$ according to the band presentation for $K$. This produces a two-component link, with $B$ becoming $K$ and $R$ becoming a 0-framed unknot $c$ linked with $K$. We adjust the framing of $B$ to $n$. If we add a meridian $G$ to $R$ and slide $B$ back over $R$, then we obtain a diagram fitting into Picirillo's construction.

To obtain the knot $K_n'$, we begin with the two component link $B\cup R$ described above, where $B$ is $K$ with framing $n$ and $R$ is a 0-framed unknot. We add a 0-framed meridian $G$ to $K$, and we note that if $B$ has nonzero framing, then by a slam dunk, we can change the framing of $B$ to 0 if we add a meridian $P$ to $B$ with framing $-1/n$. We slide $B$ over $R$, and we isotope the diagram until $B$ and $R$ are both unknots which cross each other twice, i.e., $B$ and $R$ form a Hopf link if $G$ is removed. We then slide $G$ over $B$ until $G$ is no longer linked with $R$. At this stage, we can remove $R$ using a ``lightbulb trick." This leaves us with a two-component link $G\cup P$. The component $G$ is the knot $K_0'$ with the same 0-surgery as $K$, and $P$ is an unknot $c'$ linked with $K_0'$. We obtain $K_n'$ by an $n$-Rolfsen twist of $G\cup P$, i.e., we twist $K_0'$ along $c'$, $n$ times.


\subsection{Twist families of knots}

We follow Kouno, Motegi, and Shibuya's treatment of the operation of twisting a knot along a solid torus. Let $K$ be an unoriented smooth knot in the oriented 3-sphere $S^3$ and $V$ a solid torus with a preferred framing that contains $K$ in its interior. Define an orientation-preserving homeomorphism $f_n$ of $V$ such that $f_n(m)=m$ and  $f_n(l)=l+nm$ in $H_1(\partial V)$ where $(m, l)$ is a preferred meridian-longitude pair of $V$. This twisting homeomorphism $f_n$ defines a new knot $f_n(K)$ in $S^3$, the result of twisting $K$ along $V$. Write $K_{V,n} := f_n(K)$. The wrapping number $w_V(K)$ of $K$ in $V$ is defined to be the minimal intersection number of $K$ with a meridional disc in $V$.


From a two-component link $K\cup c$ with $c$ an unknot linked with $K$, we can obtain a new knot $K_n := K_{V,n}$ by taking $V$ to be a solid torus with meridian $c$ that contains $K$ in its interior. In particular, we obtain $K_n$ from $K\cup c$ by twisting $n$ times along $c$. Note that $w_V(K)=1$ if and only if $c$ is a meridian to $K$. Moreover, $K_0=K$. We define the \textit{twist family} with base knot $K$ and unknot $c$ to be the set $\{K_n \,|\, n \in \mathbb{Z}\}$. Note that for all $m, n\in\mathbb{Z}$,  $K_m$ is obtained from $K_n$ by an ($m-n$)-twist along $V$ \cite[Remark 2.2]{twisting-knot-types}, so any knot $K_n$ can be taken as the base of the twist family. The knots $K_n'$ described in Section 2.2 form a twist family, with base $K_0'$ and unknot $c'$. 

Kouno, Motegi, and Shibuya prove that for a twist family $\{K_{V,n}\}$ in a solid torus $V$ with $w_V(K)\geq 2$, there are at most finitely many integers $n_i$ for which $K_{V,n_i}\sim K$  \cite[Theorem 3.2]{twisting-knot-types}. Here, $K_{V,n_i}\sim K$ means that there exists a homeomorphism of $S^3$ carrying $K_{V,n_i}$ to $K$. This is a weaker condition than the existence of an orientation preserving homeomorphism of $S^3$ sending $K_{V,n_i}$ to $K$, which is equivalent to the existence of an ambient isotopy between $K_{V,n_i}$ and $K$ \cite[Introduction]{twisting-knot-types}. Using this fact and translating the theorem into the language of the link $K\cup c$, we obtain the following:


\begin{theorem}[Kouno-Motegi-Shibuya] Let $K$ be a knot in $S^3$ and $c$ an unknot linked with $K$. Let $\{K_n\}$ be the twist family with base knot $K$ and unknot $c$. If $c$ is not a meridian to $K$, then there are at most finitely many integers $n$ such that $K_n\simeq K$. 
\end{theorem}

Here, the relation $\simeq$ denotes ambient isotopy of knots in $S^3$.

\subsection{Proof of Theorem~\ref{thm:unknotting-one}} We show that under the hypotheses on $K$, $K_n'\simeq K$ for at most finitely many $n$. If $K_n'\not\simeq K$ for all $n$, the conclusion holds. If $K_N'\simeq K$ for some $N$, then we regard $K_N'$ as the base knot of the family $\{K_n'\}$. Note that $c$ is not a meridian to $K_N'$ as long as it is not a meridian to $K_0'$. To see this, note that if $c'$ is a meridian to $K_N'$, then twisting through $c'$ does not change $K_N'$; hence by twisting $-N$ times, we see that $K_N'\simeq K_0'$, and $c'$ is a meridian to $K_0'$. It remains to show that $c'$ is not a meridian to $K_0'$, for then Theorem 2.5 applies to the twist family $\{K_n'\}$ to show that $K_n'\simeq K_N'\simeq K$ for at most finitely many $n$. 

Recall from Theorem~\ref{thm:twd} that since $K$ is not a twisted Whitehead double, in any band presentation for $K$, the band must cross one component of the Hopf link. The following lemma therefore proves that in the link $K\cup c$ appearing in Piccirillo's construction, the unknot $c$ is not a meridian to $K$.

\begin{lemma}
	Let $R\cup B$ be a Hopf link, and consider a handle slide of $R$ over $B$ which leaves $R$ a meridian to $B$. Then there is an equivalent handle slide of $R$ over $B$ along a band which does not cross either $R$ or $B$.
\end{lemma}
\begin{proof}
	TBD
\end{proof}

We now appeal to a lemma of Baker and Motegi \cite[Lemma 2.4]{baker-motegi}, with notation adapted:

\begin{lemma}[Baker-Motegi, 2018]
	Let $K_0'\cup c'$ be a two-component link in $S^3$ such that $c'$ is a meridian of $K_0'$. Then $(0,0)$-surgery on $K_0'\cup c'$ results in $S^3$ with its surgery dual link $c\cup K$, for which $c$ is a meridian to $K$.
\end{lemma}

\begin{corollary}
	If $c$ is not a meridian to $K$, then $c'$ is not a meridian to $K_0'$.
\end{corollary}

\begin{proof}
	We know that $c'$ and $K_0'$ are surgery duals to $K$ and $c$ respectively. Show that it follows that $c$ and $K$ are dual to $K_0'$ and $c'$ respectively.
\end{proof}
This completes the proof of the theorem.

\section{Knots with low crossing number}

\subsection{Possible extensions of Theorem 1.1} \kh{[Note that $u(K)=1$ is not necessary for Piccirillo's construction. We only need to be able to produce a link $R\cup B$ (with $B\simeq K$ and $R$ an unknot) from handle slides on a Hopf link. Explain this procedure for knots with $u(K)>1$.]}

\subsection{Hyperbolic Dehn surgery} \kh{[Recall and discuss the relevant theorems used in our approach.]}

\subsection{Proof of Theorem~\ref{thm:low-crossing}}

To prove Theorem~\ref{thm:low-crossing}, we need to rule out infinitely many characterizing slopes for each of the knots $K$ we are interested in. In general, this is a two-part algorithm:

\begin{enumerate}
	\item We use the theorems of hyperbolic Dehn surgery to find a bound $N \in \zz_{>0}$ such that any slope $n \in \zz$ is not characterizing if $|n|>N$.
	\item For the $2 N + 1$ remaining cases, we directly examine a knot $K_{n}'$ with the same $n$-surgery as $K$. If $K_{n}' \neq K$, we know that $n$ is noncharacterizing.
\end{enumerate}

\kh{[First part, based on hyperbolic Dehn surgery theorems. Explain how we found N.]}

Finding this bound and examining the volumes of each of $2 N + 1$ knots in the twist family of each knot $K$ is a tedious computational problem. It was convenient for us to let a computer program run this proof given minimal input.

\subsubsection{Code and Data}
We produced a framed link $L = K \cup R \cup c$ for each of the knots $K$, where $R$ and $c$ are unknotted components and $c$ is a meridian to $R$. This link is such that, if the framing on $R$ is zero and the framings on $K$ and $c$ are any integer numbers, an equivalent surgery diagram to $L$ is a link $L' = B \cup R \cup G$ that satisfies Piccirilo's theorem, with $K$ turning into $B$ and $c$ turning into $G$.

Any link can be described by a partitioned sequence of even numbers called its Dowker-Thistlethwaite  code, or DT code. This is provided by the software SnapPy. In the case of a link $L$ fitting the description above, we use the following lemma:

\begin{lemma}
	Up to isotopy, $L$ can be uniquely described by its DT code.
\end{lemma}
\begin{proof}
	TBD
\end{proof}

The script \texttt{integral\char`_slopes.py}, given the DT code of a link $L$ satisfying the conditions above, outputs a list of possible integral characteristic slopes for any knot $K$. It relies on writing a text file, running it on SnapPy, saving the output, and interpreting it repeatedly. There are certain system requirements explicit on the code. The script consists of the following steps:

\begin{enumerate}
	\item From the DT code, the program writes the file \texttt{[INFO]K.py}. This file asks for the volume of the knot $K$, for the volume of the manifold $Z = S_0^3(R)$ obtained by zero-surgery filling the 1-handle component, and for the length of the Seifert longitude in $Z$.
	\item SnapPy opens, runs, and saves this file, producing a first output file named \texttt{[OUT][INFO]K.py}.
	\item The program extracts this information and calculates the bound $N$ using the theorems of hyperbolic Dehn surgery.
	\item The program writes the file \texttt{[TEST]K.py}. This file asks for the volume of each knot $K_{n}'$ in the twist family of $K$, whose exterior is easily obtained by doing $n$-surgery on the first component of the manifold $Z$, for all $n$ such that $|n| \leq N$.
	\item SnapPy opens, runs, and saves this file, producing a second output file named \texttt{[OUT][TEST]K.py}.
	\item (\textit{Optional}) The program verifies if this last file contains all $2 N + 1$ cases. Because the run time in SnapPy is sometimes unpredictable, this step is recommended although the proof may be correct and faster without it.
	\item The program identifies the knots whose volume is less than or equal (up to approximations) to the volume of the knot $K$. Those cases are singled out in the output file and, if possible, identified by SnapPy.
\end{enumerate}

All files corresponding to the knots listed in Theorem~\ref{thm:low-crossing} can be found on \kh{Kyle's website}. These files prove the theorem.

\begin{proof} We produced the link $L$ aforementioned for all knots $K$ mentioned in Theorem~\ref{thm:low-crossing}. All the DT codes are included in the file \texttt{DT\char`_List.txt}, which is correctly formatted for the script \texttt{integral\char`_slopes.py}. Running it outputs the \texttt{.py} files labeled \texttt{[OUT]}, and text that is reproduced in the file \texttt{results.txt}. From reading this file, we see that:
	\begin{enumerate}[label=\normalfont (\alph*)]
		\item For all knots $K$ of crossing number at most 10 with unknotting number $u(K) = 1$ and not twist knots, the program indicates that only $n = 2$ or $n = -2$ could be characterizing.
		\kh{\item For $K$ one of the knots $8_4$, $8_6$, $8_{10}$, or $8_{12}$, the program indicates that all knots $K_{n}'$ have different volume than $K$, so no integral slopes can be characterizing.}
		\item For the knot $8_1$, the program indicates that only $n = 0$ could be characterizing.
	\end{enumerate}
\end{proof}

\section{Additional results}

\kh{[Any additional findings, including HFK or Khovanov homology findings, or enhanced sliceness obstructions using Piccirillo's technique, can go here.]}


\bibliographystyle{alpha}
\bibliography{biblio}

\end{document}
\documentclass[11pt,usenames,dvipsnames,reqno]{amsart} 
%\usepackage[margin=1.34in]{geometry}
\usepackage{xypic}
\usepackage{amsmath}
\usepackage{amssymb}
\usepackage{amsthm}
%\usepackage[notref,notcite]{showkeys}
\usepackage[colorlinks=true,linkcolor=NavyBlue,citecolor=NavyBlue,anchorcolor=NavyBlue,urlcolor=NavyBlue]{hyperref}
\usepackage{graphicx}
\usepackage{enumitem}
\usepackage{tikz}

%Shortcuts
\newcommand{\cc}{\mathbb{C}}
\newcommand{\zz}{\mathbb{Z}}
\newcommand{\rr}{\mathbb{R}}
\renewcommand{\sl}{\mathrm{sl}}
\newcommand{\tb}{\mathrm{tb}}
\newcommand{\lk}{\mathrm{lk}}
\newcommand{\st}{{\mathrm{st}}}
\newcommand{\id}{\operatorname{id}}
\newcommand{\pr}{\operatorname{pr}}
\newcommand{\Symp}{\operatorname{Symp}}
\newcommand{\Mod}{\operatorname{Mod}}
\newcommand{\Diff}{\operatorname{Diff}}
\newcommand{\km}{{\normalfont{(KM)}}}
\newcommand{\vol}{\operatorname{vol}}


%Environments
\newtheoremstyle{thm}{15 pt}{10 pt}{\itshape}{}{\bfseries}{.}{.5em}{}
\newtheorem{thm-main}{Theorem}
\newtheorem{theorem}{Theorem}
\numberwithin{theorem}{section}
\newtheorem{corollary}[theorem]{Corollary}
\newtheorem{question}[theorem]{Question}
\newtheorem{lemma}[theorem]{Lemma}
\newtheorem{conjecture}[theorem]{Conjecture}
\newtheorem{proposition}[theorem]{Proposition}
\newtheorem*{thm*}{Theorem}
\newtheorem*{cor*}{Corollary}

\newtheoremstyle{ex}{10 pt}{10 pt}{\normalfont}{}{\bfseries}{.}{.5em}{}
\theoremstyle{ex}
\newtheorem{example}[theorem]{Example}
\newtheorem{definition}[theorem]{Definition}

\newtheoremstyle{rem}{10 pt}{10 pt}{\normalfont}{}{\it}{.}{.5em}{}
\theoremstyle{rem}
\newtheorem{remark}[theorem]{Remark}

\usepackage{xcolor}
\def\kh#1{\textcolor{Blue}{#1}}
\begin{document}

%%%%%%%%%%%%%%%%%%%%%%%%
%                     End of preamble                       %
%%%%%%%%%%%%%%%%%%%%%%%%


\title{Integer characterizing slopes and unknotting numbers}

\author[Agostini, Chen, Daemi, Hayden, Serio, Wang, A. Wu, and K. Wu]{Gabriel Agostini, Sophia Chen, Aliakbar Daemi, Kyle Hayden, Christian Serio, Cecilia Wang, Anton Wu, and Kexin Wu}



\begin{abstract} 
\kh{[Write one!]}
\end{abstract}


\maketitle

\section{Introduction}\label{sec:intro}

Given a knot $K\subset S^3$, we say that $p/q$  is a {\it characterizing slope} for $K$ if the oriented homeomorphism type of the manifold obtained by $p/q$-surgery on $K$ determines $K$ uniquely up to isotopy. Denote isotopy by $\simeq$ and orientation-preserving homeomorphism by $\cong$.  
It was a long-standing conjecture of Gordon before it was eventually proved by Kronheimer, Mr\'{o}wka, Ozsv\'{a}th, and Szab\'{o} that every slope is characterizing for the trefoil and the figure eight knot. In 2018, Picirillo developed a construction that yields orientation-preserving homeomorphic manifolds after $n$-surgery on a knot $K$ in $S^3$ and inspired us to find knots $K'_n$ that share the same $n$-surgery as $K$. She adopted a constructivist approach to produce $K_0'$ while Baker and Motegi defined $K'_0$ more theoretically using surgical duality.(Purpose of the paper)
*put an introduction between 1.2 and 1.3 

(Describe Picirillo’s construction in an appendix) 


\kh{[Introduction with some motivation and introduction of terms like ``characterizing slopes.'' See the papers by Ni and Zhang and by McCoy and by Lackenby on characterizing slopes for ideas to steal. Mention Piccirillo's results for unknotting number one and using this to show that the Conway knot isn't slice. Mention that McCoy's work \cite{mccoy:hyperbolic}, which shows that a hyperbolic knot has only finitely many non-characterizing slopes $p/q$ with $|q| \geq 3$. In a sense, this implies that ``most'' slopes $p/q$ are characterizing for any given hyperbolic knot $K$: The  probability that a randomly chosen slope $p/q$ is characterizing approaches 1 as $|p|+|q|\to \infty$.]}

\begin{theorem}\label{thm:unknotting-one} If a knot $K$ has unknotting number $u(K)=1$ and is not a twisted Whitehead double, then $K$ has at most finitely many integer characterizing slopes.
\end{theorem}

We worked our way towards Theorem ~\ref{thm:unknotting-one} by first manually finding knots $K_{0}'$ with the same $0$-surgery as a given knot $K$ with $u(K)=1$ inspired by Piccirillo's construction and then devising a mechanism to produce all $K_n'$ that shares the same $n$-surgery with $K$. Beginning with the former process, we found the following result for knots with low crossing number:

\begin{theorem}\label{thm:low-crossing} For knots $K$ with crossing number $c(K) \leq 10$:
	\begin{enumerate}[label=\normalfont \bf (\alph*)]
		\item If $K$ has unknotting number $u(K)=1$ and $K$ is not a twist knot, then $K$ has at most one integer characterizing slope, namely $\pm 2$.
		\item If $K$ is one of the knots $8_4$, $8_6$, $8_{10}$, or $8_{12}$, then $K$ has $u(K)>1$ and has no integer characterizing slopes.
		\item If $K$ is the twist knot $8_1$, then $K$ has at most one integer characterizing slope, namely $0$.
	\end{enumerate}
\end{theorem}

This theorem specifies the number of integer characterizing slopes for several knots with $c(K) \leq 10$. It is possible that the bound on $c(K)$ can be increased, especially for knots with unknotting number $u(K) = 1$. Similarly, the list in part (b) can probably be expanded, encompassing every knot that fits Piccirilo's construction. Part (c) of this theorem also shows that the assumption in Theorem ~\ref{thm:unknotting-one} that $K$ is not a twisted Whitehead double is not a necessary condition for the conclusion to hold. Moreover, it was proven \kh{by Ozsv\'{a}th and Szab\'{o} 2006, maybe cite}  that every slope is characterizing for the trefoil and the figure-eight knot. Part (c) shows that this fact does not generalize to twist knots with higher crossing numbers.

Theorem 1.2 affirmatively answers a question posed by Baker and Motegi, who ask if there are knots with crossing number less than eight that have infinitely many non-characterizing slopes \cite[Question~1.7]{baker-motegi}. In fact, Theorem 1.2 shows that all knots with crossing number less than eight and unknotting number one which are not twist knots have infinitely many non-characterizing slopes. \kh{[See if we can expand this result to be independent of $u(K)$.]}

The proof of this theorem relies in part on a computer program to calculate volumes of finitely many surgeries on each knot. We use the software SnapPy, which can triangulate hyperbolic manifolds. The script used is included on \kh{maybe put on Appendix?}, and the files produced for all knots encompassed can be found on \kh{Kyle's website}.

\begin{conjecture}[Baker] If $K$ and $K'$ are non-isotopic knots in $S^3$ which yield the same $3$-manifold $Y$ under $p/q$-surgery, then their surgery duals $\gamma$ and $\gamma'$ in $Y$ are not homotopic.
\end{conjecture}

We showed that Baker's conjecture holds for all knots $K$ with unknotting number $u(K) = 1$ and crossing number $c(K) \leq 10$ in the case of zero surgery. The files corresponding to this verification can be found on \kh{Kyle's website}.


\section{Knots with unknotting number one}\label{sec:unknotting-one}


\subsection{Banded Hopf link diagrams}\label{band} A knot $K$ with unknotting number one can be represented by a diagram that consists of a Hopf link with its two components connected by a band. We have the following result:

\begin {theorem} If $K$ has unknotting number one, then it is obtained from the Hopf link by a single band move.
\end{theorem}

To find the banded Hopf link diagram for a knot $K$ with $u(K) =1$, we find the unkotting crossing of $K$ and alternate it. We then draw a meridian adjacent to it \kh{[Diagram showing how meridian has to be parallel to crossing]} and slide $K$ over the meridian. After simplification, we obtain the banded Hopf link diagram for $K$. 

Twisted Whitehead double knots has multiple definitions. Here we use a characterization in terms of band presentations.

\begin{definition} A knot $K$ is a \textit{twisted Whitehead double} if there exists a band presentation for $K$ in which the band does not cross either component of the Hopf link.
	
Note that twist knots are twisted Whitehead doubles, but not all twisted Whitehead doubles are twist knots.

\end{definition}

\subsection{Baker and Motegi's Construction} We follow Baker and Motegi \cite{baker-motegi} in constructing knots $K_n'$ with the same $n$-surgery as a given knot $K$ satisfying certain hypotheses. We prove the following, which strengthens Theorem 1.3 in Baker-Motegi:

\begin{proposition}\label{prop 2.3}
	Let $K$ be a knot in $S^3$. Suppose we can take an unknot $c$ linked with $K$ so that $(0,0)$-surgery on $K\cup c$ yields $S^3$ and $c$ is not a meridian of $K$. Then $K$ has at most finitely many integer characterizing slopes.
\end{proposition}

\begin{proof}
Recall the definition of a surgery dual to a knot $K\subset S^3$. When we perform $p/q$-surgery on $K$, we thicken $K$ to obtain a solid torus and remove this solid torus from $S^3$ and glue in a new solid torus in a manner specified by $p/q$. The core of this new torus yields a new knot $\gamma$, the \textit{surgery dual} of $K$ in the manifold $S^3_{p/q}(K)$. The following lemma establishes the symmetry of surgical duality for knots.

\begin{lemma}
	For $K$ a knot in $S^3$, if $\gamma$ is the surgery dual to $K$ in $S^3_n(K)$ for $n\in\mathbb{Z}$ and $K'$ is the dual to $\gamma$ in the surgered manifold $Y$, then $Y\simeq S^3$ and $K\simeq K'$ in $S^3$.
\end{lemma}

\begin{proof}
	In a Kirby diagram for $n$-surgery on $K$, we can represent $\gamma$ by a small meridian to $K$ and the dual $K'$ of $\gamma$ by a meridian to $\gamma$. Note that $0$-surgery on $\gamma$ along with $n$-surgery on $K$ in $K\cup\gamma$ yields $S^3$ by a slam dunk and thus $Y \simeq S^3$. In other words, doing $n$-surgery on $K$ in $S^3$ with dual $\gamma$ and then performing a 0-surgery $\gamma$ amounts to removing a solid torus from $S^3$ and gluing the same torus back. Note that $Y$ here is specific to the $\gamma$ as the dual of $n$-surgery on $K$, and not to be viewed stripped off this context (i.e. we distinguish $Y$ from $S^3_0(P)$ where the latter is the manifold that we obtain from doing 0-surgery on an unknot $P$ alone). So it makes sense to compare $K$ and $K'$ since they live in the same manifold $S^3$. We can slide $K'$ over $K$ in the surgery diagram since handle slides preserve isotopies and isotope $K'$ off of $\gamma$ to remove the cancelling pair $K\cup \gamma$. As a result, we are left with $K'$ that is isotopic to $K$. 
\end{proof}

Since a link is a disjoint union of knots, surgery on a framed link $L =\bigcup\limits_{i=1}^{N} L_i$ is by removing a disjoint union of solid tori and gluing in a disjoint union of new solid tori corresponding to surgery coefficients. Then we can appeal to the contrapositive of a lemma of Baker-Motegi \cite[Lemma 2.4]{baker-motegi} with $c$ (and $K$ resp.) as the dual to $K'$ (and $c'$ resp.) in $S^3$, since if (0, 0)-surgery on $K \cup c$ yields $S^3$, then (0, 0)-surgery on $c' \cup K'$ yields $S^3$ since 0-surgery on either $c'$ or $K'$ amounts to removing a solid tori from $S^3$ and gluing the same solid tori back by the above lemma.

\begin{lemma}[Baker-Motegi]
	Let $K'\cup c'$ be a two-component link in $S^3$ such that $c'$ is a meridian of $K'$. Then $(0,0)$-surgery on $K'\cup c'$ results in $S^3$ with its surgery dual link $c\cup K$, for which $c$ is a meridian of $K$.
\end{lemma}
	
We cite one more result of Baker-Motegi \cite[Theorem 2.1]{baker-motegi}, which we state as a lemma:

\begin{lemma}[Baker-Motegi]
	Let $K'\cup c'$ be a link in $S^3$ such that $c'$ is unknotted. Suppose that $(0,0)$-surgery on $K'\cup c'$ results in $S^3$. Let $K$ be the knot in $S^3$ which is surgery dual to the image of $c'$ in the surgered $S^3$, and let $K_n'$ be the knot obtained from $K'$ by twisting $n$ times along $c'$. In particular, $K_0'=K'$. Then $S^3_n(K)\cong S^3_n(K_n')$ for all integers $n$. Moreover, if $c'$ is not a meridian to $K'$, then $K\not\simeq K_n'$ for all but finitely many $n$.
\end{lemma}

We are now equipped to prove Proposition 2.3. Let $c'$ and $K'$ be the surgery duals to $K$ and $c$ respectively. Then by the contrapositive of Lemma 2.5, $c'$ is not a meridian to $K'$ since $c$ is not a meridian to $K$ by hypothesis of the proposition. It follows from Lemma 2.6 that all but finitely many of the knots $K_n'$ sharing the same $n$-surgery with $K$ are not isotopic to $K$, (i.e., all but finitely many integers are non-characterizing slopes for $K$).
\end{proof}


\subsection{Proof of Theorem~\ref{thm:unknotting-one}} We show that if $K$ has $u(K)=1$ and is not a twisted Whitehead double, then we can find an unknot $c$ as in Proposition~\ref{prop 2.3}. 

Consider the band presentation for $K$ as described in Section~\ref{band}. Start with a (0, 0)-framed Hopf link $R\cup B$ and slide $R$ over $B$ once according to the band presentation. Now $R$ becomes $K$ with $B$ an unknot $c$ linked with $K$. To show that (0, 0)-surgery on $K\cup c$ yields $S^3$, we have the following lemma. 

\begin{lemma}
	Performing $(0,0)$-surgery on the link $K\cup c$ yields $S^3$.
\end{lemma}

\begin{proof}
	We obtain $K\cup c$ from a (0, 0)-framed Hopf link $R \cup B$ by sliding $R$ over $B$ once, after which $R$ becomes $K$ with framing $\pm 2$ and $B$ as a 0-framed unknot $c$. If we adjust the initial framing of $R$ to $\mp 2$, then the handle slide yields the same link $K\cup c$ with both $K$ and $c$ 0-framed. Thus $(0,0)$-surgery on $K\cup c$ is homeomorphic to $(\mp 2,0)$ surgery on $R\cup B$ which yields $S^3$ since $(n,0)$-surgery on a Hopf link yields $S^3$ for any integer $n$, .
\end{proof}

For $K$ not a twisted Whitehead double, the band must cross one component of the Hopf link in all of its band presentations. The following lemma therefore proves that $c$ is not a meridian to $K$ with $c$, $K$ in the above construction.

\begin{lemma}
	Let $R\cup B$ be a Hopf link, and consider a handle slide of $R$ over $B$ which leaves $R$ a meridian to $B$. Then there is an equivalent handle slide of $R$ over $B$ along a band which does not cross either $R$ or $B$.
\end{lemma}
\begin{proof}
	TBD
\end{proof}

Theorem 1.1 now follows from Proposition 2.3.

\section{Knots with low crossing number}

\subsection{Hyperbolic Dehn surgery} \kh{[Recall and discuss the relevant theorems used in our approach.]}

\subsection{Proof of Theorem~\ref{thm:low-crossing}}

To prove Theorem~\ref{thm:low-crossing}, we need to rule out infinitely many characterizing slopes for each of the knots $K$ we are interested in. In general, this is a two-part algorithm:

\begin{enumerate}
	\item We use the theorems of hyperbolic Dehn surgery to find a bound $N \in \zz_{>0}$ such that any slope $n \in \zz$ is not characterizing if $|n|>N$.
	\item For the $2 N + 1$ remaining cases, we directly examine a knot $K_{n}'$ with the same $n$-surgery as $K$. If $K_{n}' \neq K$, we know that $n$ is noncharacterizing.
\end{enumerate}

\kh{[First part, based on hyperbolic Dehn surgery theorems. Explain how we found N.]}

Finding this bound and examining the volumes of each of $2 N + 1$ knots in the twist family of each knot $K$ is a tedious computational problem. It was convenient for us to let a computer program run this proof given minimal input.

\subsubsection{Code and Data}\label{sec:code}
We produced a framed link $L = K \cup R \cup c$ for each of the knots $K$, where $R$ and $c$ are unknotted components and $c$ is a meridian to $R$. This link is such that, if the framing on $R$ is zero and the framings on $K$ and $c$ are any integer numbers, an equivalent surgery diagram to $L$ is a link $L' = B \cup R \cup G$ that satisfies Piccirilo's theorem, with $K$ turning into $B$ and $c$ turning into $G$.

Any link can be described by a partitioned sequence of even numbers called its Dowker-Thistlethwaite  code, or DT code. This is provided by the software SnapPy. In the case of a framed link $L$ fitting the description above, we use the following lemma:

\begin{lemma}
	Up to isotopy, $L$ can be uniquely described by its DT code.
\end{lemma}
\begin{proof}
	This lemma is a direct application of Doll and Hoste's Theorem 1.2 \kh{[CITE]} to our link $L$. As defined, $L$ will always be a nonsplit link whose orientation is irrelevant for the purposes of integral surgery. $R$ and $c$ are two unknots, so $L$ is a prime link if and only if $K$ is a prime knot. That is the single case we are interested in, since only prime knots are classified along with their invariants and diagrams. Thus, by Doll and Hoste, minimal projections of $L$ are in one-to-one correspondence with valid DT codes.
	Finally, there is a single solution for the question of which of the components is $K$, $R$, and $c$. We use that $c$ is strictly a meridian to $R$, so its only crossings are the two they share. Since the entries on the DT code come from crossings either within a single component or between two different ones, we notice that the code corresponding to the component $c$ will have length 1 (since, by definition of a DT code, only even labeled crossings appear). $R$ also does not have crossings with itself, but it has crossings with $R$ and $c$, so its code has length greater than 1. Moreover, since $K$ is not an unknot, $c(K) \geq 3$, so the code corresponding to $K$ will always be longer than the code corresponding to $R$. Thus, it is clear that $l(K) > l(R) > l(c) = 1$, where $l(X)$ is the code length of the component $X$.
\end{proof}

The script \texttt{integral\char`_slopes.py}, given the DT code of a link $L$ satisfying the conditions above, outputs a list of possible integral characteristic slopes for any knot $K$. It relies on writing a text file, running it on SnapPy, saving the output, and interpreting it repeatedly. There are certain system requirements explicit on the code. The script consists of the following steps:

\begin{enumerate}
	\item From the DT code, the program writes the file \texttt{K [INFO].py}. This file asks for the volume of the knot $K$, for the volume of the manifold $Z = S_0^3(R)$ obtained by zero-surgery filling the 1-handle component, and for the length of the Seifert longitude in $Z$.
	\item SnapPy opens, runs, and saves this file, producing a first output file named \texttt{[OUT] K [INFO].py}.
	\item The program extracts this information and calculates the bound $N$ using the theorems of hyperbolic Dehn surgery.
	\item The program writes the file \texttt{K [TEST].py}. This file asks for the volume of each knot $K_{n}'$ in the twist family of $K$, whose exterior is easily obtained by doing $n$-surgery on the first component of the manifold $Z$, for all $n$ such that $|n| \leq N$.
	\item SnapPy opens, runs, and saves this file, producing a second output file named \texttt{[OUT] K [TEST].py}.
	\item The program verifies if this last file contains all $2 N + 1$ cases. This step is included because the run time in SnapPy is sometimes unpredictable.
	\item The program identifies the knots whose volume is less than or equal (up to approximations) to the volume of the knot $K$. Those cases are singled out in the output file and, if possible, identified by SnapPy.
\end{enumerate}

All files corresponding to the knots listed in Theorem~\ref{thm:low-crossing} can be found on \kh{Kyle's website}. These files prove the theorem.

\begin{proof} We produced the link $L$ aforementioned for all knots $K$ mentioned in Theorem~\ref{thm:low-crossing}. All the DT codes are included in the file \texttt{DT\char`_List.txt}, which is correctly formatted for the script \texttt{integral\char`_slopes.py}. Running it outputs the \texttt{.py} files labeled \texttt{[OUT]}, and text that is reproduced in the file \texttt{output.txt}. From reading this file, we see that:
	\begin{enumerate}[label=\normalfont (\alph*)]
		\item For all knots $K$ of crossing number at most 10 with unknotting number $u(K) = 1$ and not twist knots, the program indicates that only $n = 2$ or $n = -2$ could be characterizing.
		\kh{\item For $K$ one of the knots $8_4$, $8_6$, $8_{10}$, or $8_{12}$, the program indicates that all knots $K_{n}'$ have different volume than $K$, so no integral slopes can be characterizing.}
		\item For the knot $8_1$, the program indicates that only $n = 0$ could be characterizing.
	\end{enumerate}
\end{proof}

\subsection{Possible extensions of Theorem 1.1}

Theorem~\ref{thm:low-crossing} is stronger than Theorem~\ref{thm:unknotting-one} for low-crossing number knots, since it is valid for knots $K$ such that the unknotting number $u(K) > 1$ and even some Twisted Whitehead Doubles. It suggests that the previous theorem can be extended to encompass a large number of knots than those with $u(K)=1$.

To use Piccirillo's construction, the original band presentation is not necessary. Any link $L' = K\cup R$, where $R$ is a one-handled unknot can receive a meridian $c$ to become the link $L$ described in section~\ref{sec:code}. This link fits Piccirilo's construction, and will yield a non-trivial twist family of knots $K_{n}'$ if and only if $R$ is not a meridian to $K$. As proved, the banded Hopf Link presentation will always yield such $L'$ through a single handle slide, but we have encountered diagrams for this link that do not depend on a band presentation.

We prove parts (b) and (c) of Theorem~\ref{thm:low-crossing} using these diagrams. We could not find an algorithm to produce them, but a few strategies were recurrently successful:

\begin{enumerate}
	\item Start with a simple banded Hopf link diagram, then change the framing on the band and perform the handle slide.
	\item Start with a simple banded Hopf link diagram for a knot $K$ and perform the usual handle slide. Then, perform a second handle slide of $K$ over $R$ on a region where $K$ is not parallel to $R$. Sometimes, different framings on the second handle slide successfully yielded different links fitting Piccirilo's construction.
	\item Start from a simple diagram $B \cup R \cup G$ that fits Piccirilo's construction. Then, repeatedly slide $B$ over $R$ until it comes off of $G$.
\end{enumerate}

Such strategies take diagrams for knots we previously find and modify them in order to find analogous for other knots. Whether the two knots related by these operations have any relationship is unknown. If real, such relationship could lead to an algorithm that produces a link $L$ for $J$ with $u(J) > 1$ given the link $L$ for $K$ with $u(K) = 1$. Moreover, using the third strategy we were even successful to find a link $L$ that fits Piccirilo's construction for the twist knot $8_1$.

We now pose a question that would allow us to generalize Theorem~\ref{thm:unknotting-one} for more knots, dropping the assumptions on the unknotting number and the Twisted Whitehead classification.

\kh{[Not sure how to phrase this but I think it could be a good open question from our paper.]}
\begin{question}
	For what knots $K$ can a link $L' = K \cup R$, where $R$ is an unknot that fails to be a meridian of $K$, be produced through handle slides on a Hopf Link?
\end{question}

\section{Additional results}

\kh{[Any additional findings, including HFK or Khovanov homology findings, or enhanced sliceness obstructions using Piccirillo's technique, can go here.]}


\bibliographystyle{alpha}
\bibliography{biblio}

\end{document}
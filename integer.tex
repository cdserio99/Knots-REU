\documentclass[11pt,usenames,dvipsnames,reqno]{amsart} 
%\usepackage[margin=1.34in]{geometry}
\usepackage{xypic}
\usepackage{amsmath}
\usepackage{amssymb}
\usepackage{amsthm}
%\usepackage[notref,notcite]{showkeys}
\usepackage[colorlinks=true,linkcolor=NavyBlue,citecolor=NavyBlue,anchorcolor=NavyBlue,urlcolor=NavyBlue]{hyperref}
\usepackage{graphicx}
\usepackage{enumitem}
\usepackage{tikz}

%Shortcuts
\newcommand{\cc}{\mathbb{C}}
\newcommand{\zz}{\mathbb{Z}}
\newcommand{\rr}{\mathbb{R}}
\renewcommand{\sl}{\mathrm{sl}}
\newcommand{\tb}{\mathrm{tb}}
\newcommand{\lk}{\mathrm{lk}}
\newcommand{\st}{{\mathrm{st}}}
\newcommand{\id}{\operatorname{id}}
\newcommand{\pr}{\operatorname{pr}}
\newcommand{\Symp}{\operatorname{Symp}}
\newcommand{\Mod}{\operatorname{Mod}}
\newcommand{\Diff}{\operatorname{Diff}}
\newcommand{\km}{{\normalfont{(KM)}}}
\newcommand{\vol}{\operatorname{vol}}
\newcommand{\defeq}{\mathrel{\vcenter{\baselineskip0.5ex\lineskiplimit0pt\hbox{\scriptsize.}\hbox{\scriptsize.}}}=}
\newcommand{\hs}{\hspace{0.05em}} % half-space


%Environments
\newtheoremstyle{thm}{15 pt}{10 pt}{\itshape}{}{\bfseries}{.}{.5em}{}
\newtheorem{thm-main}{Theorem}
\newtheorem{theorem}{Theorem}
\numberwithin{theorem}{section}
\newtheorem{corollary}[theorem]{Corollary}
\newtheorem{question}[theorem]{Question}
\newtheorem{lemma}[theorem]{Lemma}
\newtheorem{conjecture}[theorem]{Conjecture}
\newtheorem{proposition}[theorem]{Proposition}
\newtheorem*{thm*}{Theorem}
\newtheorem*{cor*}{Corollary}

\newtheoremstyle{ex}{10 pt}{10 pt}{\normalfont}{}{\bfseries}{.}{.5em}{}
\theoremstyle{ex}
\newtheorem{example}[theorem]{Example}
\newtheorem{definition}[theorem]{Definition}

\newtheoremstyle{rem}{10 pt}{10 pt}{\normalfont}{}{\it}{.}{.5em}{}
\theoremstyle{rem}
\newtheorem{remark}[theorem]{Remark}

\usepackage{xcolor}
\def\kh#1{\textcolor{Blue}{#1}}
\begin{document}

%%%%%%%%%%%%%%%%%%%%%%%%
%                     End of preamble                       %
%%%%%%%%%%%%%%%%%%%%%%%%


\title{Integer characterizing slopes and unknotting numbers}

\author[Agostini, Chen, Daemi, Hayden, Serio, Wang, A. Wu, and K. Wu]{Gabriel Agostini, Sophia Chen, Aliakbar Daemi, Kyle Hayden, Christian Serio, Cecilia Wang, Anton Wu, and Kexin Wu}



\begin{abstract} 
\kh{[Write one!]}
\end{abstract}


\maketitle

\section{Introduction}\label{sec:intro}

Given a knot $K\subset S^3$, we say that an extended rational number $p/q\in\mathbb{Q}\cup\{\infty\}$ is a {\it characterizing slope} for $K$ if the oriented homeomorphism type of the manifold obtained by $p/q$-surgery on $K$ determines $K$ uniquely up to ambient isotopy. In this paper, we use the symbols $\simeq$ to denote ambient isotopy (which we refer to simply as isotopy) and $\cong$ to denote orientation-preserving homeomorphism. 

In 2018, Duncan McCoy proved that any given hyperbolic knot $K$ has only finitely many non-characterizing slopes $p/q$ with $|q|\geq 3$ \cite{mccoy:hyperbolic}. Thus the probability that a randomly chosen slope $p/q$ is characterizing for $K$ approaches 1 as $|p|+|q|\to\infty$, and almost all surgery slopes are characterizing for such a knot $K$. By contrast, we show that for a large majority of knots with unknotting number one and for many knots with low crossing number, all but finitely many \textit{integer} slopes are non-characterizing.

In Section 2, we strengthen a result of Baker and Motegi, who found conditions which ensure that a knot $K$ has infinitely many integer non-characterizing slopes \cite{baker-motegi}. In Section 3, we describe an algorithm for determining the integer characterizing slopes of a given knot $K$ satisfying certain conditions. The primary results proven are the following.

\begin{theorem}\label{thm:unknotting-one} If a knot $K$ has unknotting number $u(K)=1$ and is not a twisted Whitehead double, then $K$ has at most finitely many integer characterizing slopes.
\end{theorem}

This theorem provides concrete conditions under which the strengthened result of Baker-Motegi applies. Twisted Whitehead doubles are defined in Section~\ref{band}.  We were initially led to Theorem 1.1 by the following experimental results.

\begin{theorem}\label{thm:low-crossing} For knots $K$ with crossing number $c(K) \leq 10$:
	\begin{enumerate}[label=\normalfont \bf (\alph*)]
		\item If $K$ has unknotting number $u(K)=1$ and $K$ is not a twist knot, then $K$ has at most one integer characterizing slope, namely $\pm 2$.
		\item If $K$ is one of the knots $8_4$, $8_6$, $8_{10}$, or $8_{12}$, then $K$ has $u(K)>1$ and has no integer characterizing slopes.
		\item If $K$ is the twist knot $8_1$, then $K$ has at most one integer characterizing slope, namely $0$.
	\end{enumerate}
\end{theorem}

This theorem specifies the number of integer characterizing slopes for several knots with $c(K) \leq 10$. Part (a) suggests that a stronger conclusion to Theorem 1.1 may be possible. (We remark that all twisted Whitehead doubles with crossing number less than 11 are twist knots.) The bound on $c(K)$ may be increased if the procedure described in Section~\ref{sec:code} can be better automated. The list in part (b) can also likely be expanded. Part (c) shows that the assumption in Theorem ~\ref{thm:unknotting-one} that $K$ is not a twisted Whitehead double is not a necessary condition for the conclusion to hold. It was proven \kh{by Ozsv\'{a}th and Szab\'{o} 2006, maybe cite}  that every slope is characterizing for the trefoil and the figure-eight knot. Part (c) proves that this fact does not generalize to twist knots with higher crossing numbers.

Theorem 1.2 affirmatively answers a question posed by Baker and Motegi, who ask if there are knots with crossing number less than eight that have infinitely many non-characterizing slopes \cite[Question~1.7]{baker-motegi}. In fact, Theorem 1.2 shows that all knots with crossing number less than eight and unknotting number one which are not twist knots have infinitely many non-characterizing slopes. \kh{[See if we can expand this result to be independent of $u(K)$.]}

The proof of Theorem 1.2 relies in part on a computer program to calculate volumes of finitely many surgeries on each knot. We use the software SnapPy, which can triangulate hyperbolic manifolds. The script used is included on \kh{maybe put on Appendix?}, and the files produced for all knots encompassed can be found on \kh{Kyle's website}.

\begin{conjecture}[Baker] If $K$ and $K'$ are non-isotopic knots in $S^3$ which yield the same $3$-manifold $Y$ under $p/q$-surgery, then their surgery duals $\gamma$ and $\gamma'$ in $Y$ are not homotopic.
\end{conjecture}

We showed that Baker's conjecture holds for all knots $K$ with unknotting number $u(K) = 1$ and crossing number $c(K) \leq 10$ in the case of zero surgery. The files corresponding to this verification can be found on \kh{Kyle's website}.


\section{Knots with unknotting number one}\label{sec:unknotting-one}


\subsection{Banded Hopf link diagrams}\label{band} A knot $K$ with unknotting number one can be represented by a diagram that consists of a Hopf link with its two components connected by a band. We have the following result:

\begin {theorem} If $K$ has unknotting number one, then it is obtained from the Hopf link by a single band move.
\end{theorem}

To find the banded Hopf link diagram for a knot $K$ with $u(K) =1$, we find the unkotting crossing of $K$ and alternate it. We then draw a meridian adjacent to it \kh{[Diagram showing how meridian has to be parallel to crossing]} and slide $K$ over the meridian. After simplification, we obtain the banded Hopf link diagram for $K$. 

There are multiple definitions of twisted Whitehead double knots. Here, we use a characterization in terms of band presentations.

\begin{definition} A knot $K$ is a \textit{twisted Whitehead double} if there exists a band presentation for $K$ in which the band does not cross either component of the Hopf link.
	
Note that twist knots are twisted Whitehead doubles, but not all twisted Whitehead doubles are twist knots.

\end{definition}

\subsection{Baker and Motegi's Construction} We follow Baker and Motegi \cite{baker-motegi} in constructing knots $K_n'$ with the same $n$-surgery as a given knot $K$ satisfying certain hypotheses. We prove the following, which strengthens Theorem 1.3 in Baker-Motegi:

\begin{proposition}\label{prop 2.3}
	Let $K$ be a knot in $S^3$. Suppose we can take an unknot $c$ linked with $K$ so that $(0,0)$-surgery on $K\cup c$ yields $S^3$ and $c$ is not a meridian of $K$. Then $K$ has at most finitely many integer characterizing slopes.
\end{proposition}

Recall the definition of a surgery dual. When we perform $p/q$-surgery on a knot $K\subset S^3$, we thicken $K$ to obtain a solid torus and remove this solid torus $T$ from $S^3$ and glue in a new solid torus $T'$ in a manner specified by the two integers $p$ and $q$. The core of this new torus yields a new knot $\gamma$, the \textit{surgery dual} of $K$ in the manifold $S^3_{p/q}(K)$. Similarly, since a link is a disjoint union of knots, surgery on a framed link $L =\bigcup^{N}_{i=1} L_i$ in $S^3$ is by removing a disjoint union of solid tori $\bigcup^{N}_{i=1} T_i$ from $S^3$ and gluing in a disjoint union of new solid tori $\bigcup^{N}_{i=1} T'_i$, where $T_i$ is the solid torus to be removed from $S^3$ and $T'_i$ the solid torus to be glued in when doing the corresponding surgery along the link component $L_i$. A surgery dual to $L$ is thus a link $L' = \bigcup^{N}_{i=1} L'_i$, where each component $L'_i$ is the surgery dual to $L_i$ in the surgered manifold that we obtain after doing surgery on $L$.

The following lemma establishes the symmetry of surgical duality for knots, which we generalize subsequently to an application on a link.

\begin{lemma}
	Let $K$ be a knot in $S^3$ and $n\in\mathbb{Z}$. Let $\gamma$ be the surgery dual to $K$ in $S^3_n(K)$ and $K'$ be the dual to $\gamma$ in the manifold $Y$ obtained from doing $0$-surgery on $\gamma$ in $S^3_n(K)$. Then $Y\simeq S^3$ and $K\simeq K'$ in $S^3$.
\end{lemma}

\begin{proof}
	In a Kirby diagram for $n$-surgery on $K$, we can represent $\gamma$ by a small meridian to $K$ and the dual $K'$ of $\gamma$ by a meridian to $\gamma$. Note that $(n, 0)$-surgery on the link $K \cup \gamma$ in $S^3$ yields $S^3$ by a slam dunk. This is equivalent to performing $n$-surgery on $K$ in $S^3$ first and then $0$-surgery on $\gamma$ in $S^3_n(K)$, since performing surgery on a framed link amounts to performing surgery along each of its component in any order according to our definition above. Hence $Y \simeq S^3$, and it makes sense to compare $K$ and $K'$ now that they both lie in $S^3$. We can slide $K'$ over $K$ in the surgery diagram since a handle slide represents an isotopy in the surgered manifold, and isotope $K'$ off of $\gamma$ to remove the cancelling pair $K\cup \gamma$. As a result of cancelling $K\cup \gamma$, we get $K'$ that is isotopic to $K$ in $S^3$. 
\end{proof}

Let $K \cup c$ be such that $(0, 0)$-surgery on $K \cup c$ yields $S^3$ with the surgery dual link $c' \cup K'$. Since the order in which link components are filled when doing surgery on a link does not matter, the argument on $K\cup \gamma$ in the proof of Lemma 2.4 applies to links $K\cup c'$ and $c\cup K'$ here respectively, showing that $(0,0)$-surgery on $K'\cup c'$ is $S^3$ with the dual link $c\cup K$. If further $c$ is not a meridian to $K$, we can appeal to the contrapositive of the following lemma of Baker-Motegi \cite[Lemma 2.4]{baker-motegi}. 

\begin{lemma}[Baker-Motegi]
	Let $K'\cup c'$ be a two-component link in $S^3$ such that $c'$ is a meridian of $K'$. Then $(0,0)$-surgery on $K'\cup c'$ results in $S^3$ with its surgery dual link $c\cup K$, for which $c$ is a meridian of $K$.
\end{lemma}
	
We cite one more result of Baker-Motegi \cite[Theorem 2.1]{baker-motegi}, which we state as a lemma:

\begin{lemma}[Baker-Motegi]
	Let $K'\cup c'$ be a link in $S^3$ such that $c'$ is unknotted. Suppose that $(0,0)$-surgery on $K'\cup c'$ results in $S^3$. Let $K$ be the knot in $S^3$ which is surgery dual to the image of $c'$ in the surgered $S^3$, and let $K_n'$ be the knot obtained from $K'$ by twisting $n$ times along $c'$. In particular, $K_0'=K'$. Then $S^3_n(K)\cong S^3_n(K_n')$ for all integers $n$. Moreover, if $c'$ is not a meridian to $K'$, then $K\not\simeq K_n'$ for all but finitely many $n$.
\end{lemma}


We are now equipped to prove Proposition 2.3. 

\begin{proof}
Let $c' \cup K'$ be the surgery dual link as a result of $(0, 0)$-surgery on $K \cup c$. Then by the contrapositive of Lemma 2.5, $c'$ is not a meridian to $K'$. It follows from Lemma 2.6 that all but finitely many of the knots $K_n'$ sharing the same $n$-surgery with $K$ are not isotopic to $K$, i.e., all but finitely many integers are non-characterizing slopes for $K$.
\end{proof}


\subsection{Proof of Theorem~\ref{thm:unknotting-one}} We show that if $K$ has $u(K)=1$ and is not a twisted Whitehead double, then we can find an unknot $c$ as in Proposition~\ref{prop 2.3}. 

Consider the band presentation for $K$ as described in Section~\ref{band}. Start with a (0, 0)-framed Hopf link $R\cup B$ and slide $R$ over $B$ once according to the band presentation. Now $R$ becomes $K$ with $B$ an unknot $c$ linked with $K$. To show that (0, 0)-surgery on $K\cup c$ yields $S^3$, we have the following lemma. 

\begin{lemma}
	Performing $(0,0)$-surgery on the link $K\cup c$ yields $S^3$.
\end{lemma}

\begin{proof}
	We obtain $K\cup c$ from a (0, 0)-framed Hopf link $R \cup B$ by sliding $R$ over $B$ once, after which $R$ becomes $K$ with framing $\pm 2$ and $B$ as a 0-framed unknot $c$. If we adjust the initial framing of $R$ to $\mp 2$, then the handle slide yields the same link $K\cup c$ with both $K$ and $c$ 0-framed. Thus $(0,0)$-surgery on $K\cup c$ is homeomorphic to $(\mp 2,0)$ surgery on $R\cup B$ which yields $S^3$ since $(n,0)$-surgery on a Hopf link yields $S^3$ for any integer $n$, .
\end{proof}

For $K$ not a twisted Whitehead double, the band must cross one component of the Hopf link in all of its band presentations. The following lemma therefore proves that $c$ is not a meridian to $K$ with $c$, $K$ in the above construction.

\begin{lemma}
	Let $R\cup B$ be a Hopf link, and consider a handle slide of $R$ over $B$ which leaves $R$ a meridian to $B$. Then there is an equivalent handle slide of $R$ over $B$ along a band which does not cross either $R$ or $B$.
\end{lemma}
\begin{proof}
	TBD
\end{proof}

Theorem 1.1 now follows from Proposition 2.3.

\section{Knots with low crossing number}

\subsection{Hyperbolic Dehn surgery}

Recall that a \textit{hyperbolic $3$-manifold} is a compact, orientable $3$-manifold which admits a complete hyperbolic metric, (that is, a Riemannian metric with constant negative sectional curvature); we remark that such a $3$-manifold, being compact, has a finite hyperbolic volume. A knot $K\subset S^3$ is a \textit{hyperbolic knot} if $S^3\setminus N(K)$ is a hyperbolic $3$-manifold. As before its tubular neighborhood $N(K)$ is a torus, on which simple closed curves $\gamma$ are described up to isotopy by slopes $p/q\in\mathbb{Q}\cup\{\infty\}$. The torus $\partial N(K)$ inherits a Euclidean metric, and the \textit{length} $\ell(p/q)$ of a slope $p/q$ is the length, according to this Euclidean metric, of the shortest curve on the torus $\partial N(K)$ with slope $p/q$.

Thurston's hyperbolic Dehn surgery theorem \cite[\S5.8]{thurston-notes} states that given any cusped hyperbolic $3$-manifold $M$, for each cusp there are finitely many slopes, known as `exceptional slopes', for which the $3$-manifold obtained by Dehn filling on $M$ does not inherit the hyperbolic metric. Thurston also showed \cite[p.~268]{benedetti-petronio} that if Dehn filling on $M$ yields a different hyperbolic $3$-manifold $M'$, then the volume of $M'$ is strictly smaller than the volume of $M$.

Moreover, the famous `$2\pi$' theorem due to Gromov and Thurston gives a necessary condition for a slope to be exceptional; for our purposes the theorem says the following \cite{cooper-lackenby}:

\begin{theorem}[Gromov-Thurston]\label{GromovThurston}
Let $K$ be a hyperbolic knot. If $\ell(p/q)>2\pi$, then the $(p/q)$-filling on $S^3\setminus N(K)$ is hyperbolic.
\end{theorem}

Futer, et al., \cite{futer} proved the following bound on the volume of the $(p/q)$-filling on $S^3\setminus K$:

\begin{theorem}[Futer, et al.]\label{Futer}
Let $K$ be a hyperbolic knot; let $X\defeq S^3\setminus N(K)$. If $\ell(p/q)>2\pi$, then the volume $\vol(X_{p/q})$ of the $(p/q)$-filling on $X$ satisfies
\[\vol(X_{p/q})\ge\left(1-\left(\frac{2\pi}{\ell(p/q)}\right)^2\,\right)^{3/2}\vol(X).\]
\end{theorem}

Finally, a result about slope length, found by Cooper and Lackenby \cite{cooper-lackenby}:

\begin{theorem}[Cooper-Lackenby]\label{CooperLack}
Let $X$ be a cusped hyperbolic $3$-manifold, and suppose $s_1$, $s_2$ are two slopes on a torus $T\subset \partial X$. Then
\[\ell(s_1)\hs\ell(s_2)\ge\sqrt{3}\hspace{0.2em}\Delta(s_1,\hs s_2),\]
where $\Delta(s_1,\hs s_2)$ is the minimum possible number of intersections between a curve with slope $s_1$ and a curve with slope $s_2$ on $T$.
\end{theorem}

\subsection{Proof of Theorem~\ref{thm:low-crossing}}

Let $K$ be a knot in $S^3$. We can produce a link $(K\cup R)\subset S^3$, where $R$ is an unknot which is not a meridian to $K$ and such that $(0,\hs0)$-surgery on $K\cup R$ yields $S^3$. Then, we form the link $L = K \cup R \cup c$ by adding a meridian $c$ to $R$. We see that $(n,\hs0,\hs0)$-surgery on $K \cup R \cup c$ yields $S^3_n(K)$ since $R$ and $c$ form a canceling pair. On the other hand, if we remove a tubular neighborhood of $c$ and perform $0$-surgery on $R$, then we have a hyperbolic manifold $X$ containing the surgery dual $K'$ to $K$, on which $n$-surgery yields $S^3\setminus K_n'$.

Our aim is to show that $\vol(S^3\setminus K_n')\ne\vol(S^3\setminus K)$, so that $K_n'$ is a different knot from $K$ and thus that $n$ is not a characterizing slope for $K$ for all but finitely many integers $n$. We utilize a two-part algorithm:

\begin{enumerate}
	\item We use the theorems of hyperbolic Dehn surgery to find a bound $N \in \zz_{>0}$ such that any slope $n \in \zz$ is not characterizing if $|n|>N$.
	\item For the $2 N + 1$ remaining cases, we directly examine a knot $K_{n}'$ with the same $n$-surgery as $K$. If $K_{n}' \neq K$, we know that $n$ is noncharacterizing.
\end{enumerate}

First, let $v\defeq\vol(S^3\setminus K)$, $v_n'\defeq\vol(S^3\setminus K_n')$ and $v_X\defeq\vol(X)$ represent the volumes of each manifold. We drill the neighborhood $N(K')$ from $X$ and let $\ell$ be the length of the slope $0$ on the torus $\partial N(K')$. By Theorem~\ref{thm:CooperLack}, the length ${\ell}_n$ of the slope $n$ on $\partial N(K')$ must satisfy

$${\ell}_n\ge\tfrac{1}{\ell}\hs|n|\sqrt{3}.$$

For $n$ large enough, we have ${\ell}_n>2\pi$; thus, we can apply the above bound of Futer, et al., to relate the volumes $v_X$ and $v_n'$ before and after the $n$-filling on $N(K')$:

$$v_n'\ge\left(1-\left(\frac{2\pi}{{\ell}_n}\right)^2\,\right)^{3/2} v_X.$$

Now, for $n$ large enough, we note that as $n$ increases, ${\ell}_n$ increases and thus the right-hand side increases. But, if the right-hand side is greater than $v$, then we would have $v_n'>v$ as desired. So we set $(1-(2\pi/{\ell}_n)^2)^{3/2}\cdot v_X>v$ and substitute ${\ell}_n\ge\frac{1}{\ell}\hs|n|\sqrt{3}$. Solving for $n$ gives

\begin{equation}
|n|>\frac{2\pi}{\sqrt{3}}\,\ell\hs\left(1-\left(\frac{v}{v_X}\right)^{2/3}\right)^{-1/2} = N
\end{equation}

Hence we have found $N$ such that if $|n|\ge N$, then $n$ is not a characterizing slope for $K$. This is the first step of our proof, and it only requires the existence of a link $K \cup R$ where $R$ is an unknot which is not a meridian to $K$ and such that $(0,\hs0)$-surgery on $K\cup R$ yields $S^3$. Finding this bound and examining the volumes of each of $2 N + 1$ knots in the twist family of each knot $K$ is a tedious computational problem. It was convenient for us to let a computer program run this proof given minimal input.

\subsubsection{Code and Data}\label{sec:code}
For a knot $K$, we produced a link $L = K \cup R \cup c$, where $R$ and $c$ are unknotted components and $c$ is a meridian to $R$. If $(0,\hs0)$-surgery on $K\cup R$ yields $S^3$, we would run a computer script to determine whether $K$ has at most finitely many integer characterizing slopes. We note that, if the framing on $R$ is zero and the framings on $K$ and $c$ are any integer numbers, an equivalent surgery diagram to $L$ is a link $L' = B \cup R \cup G$ that satisfies Piccirilo's theorem, with $K$ turning into $B$ and $c$ turning into $G$.

Any link can be described by a partitioned sequence of even numbers called its Dowker-Thistlethwaite  code, or DT code. This is provided and interpreted by the software SnapPy, so it was convenient for us to use the DT codes as input for our script. In the case of a link $L$ fitting the description above, we use the following lemma:

\begin{lemma}
	Up to isotopy, $L$ can be uniquely described by its DT code.
\end{lemma}
\begin{proof}
	This lemma is a direct application of Doll and Hoste's Theorem 1.2 \kh{[CITE]} to our link $L$. As defined, $L$ will always be a nonsplit link whose orientation is irrelevant for the purposes of integral surgery. $R$ and $c$ are two unknots, so $L$ is a prime link if and only if $K$ is a prime knot. That is the single case we are interested in, since only prime knots are classified along with their invariants and diagrams. Thus, by Doll and Hoste, minimal projections of $L$ are in one-to-one correspondence with valid DT codes.
	Finally, there is a single solution for the question of which of the components is $K$, $R$, and $c$. We use that $c$ is strictly a meridian to $R$, so its only crossings are the two they share. Since the entries on the DT code come from crossings either within a single component or between two different ones, we notice that the code corresponding to the component $c$ will have length 1 (since, by definition of a DT code, only even labeled crossings appear). $R$ also does not have crossings with itself, but it has crossings with $R$ and $c$, so its code has length greater than 1. Moreover, since $K$ is not an unknot, $c(K) \geq 3$, so the code corresponding to $K$ will always be longer than the code corresponding to $R$. Thus, it is clear that $\ell(K) > \ell(R) > \ell(c) = 1$, where $\ell(X)$ is the code length of the component $X$.
\end{proof}

The script \texttt{integral\char`_slopes.py}, given the DT code of a link $L$ satisfying the conditions above, outputs a list of possible integral characteristic slopes for any knot $K$. It relies on writing a text file, running it on SnapPy, saving the output, and interpreting it repeatedly. There are certain system requirements explicit on the code. The script consists of the following steps:

\begin{enumerate}
	\item From the DT code, the program writes the file \texttt{K [INFO].py}. This file asks for the volume of the knot $K$, for the volume of the manifold $Z = S_0^3(R)$ obtained by zero-surgery filling the 1-handle component, and for the length of the Seifert longitude in $Z$.
	\item SnapPy opens, runs, and saves this file, producing a first output file named \texttt{[OUT] K [INFO].py}.
	\item The program extracts this information and calculates the bound $N$ using the theorems of hyperbolic Dehn surgery.
	\item The program writes the file \texttt{K [TEST].py}. This file asks for the volume of each knot $K_{n}'$ in the twist family of $K$, whose exterior is easily obtained by doing $n$-surgery on the first component of the manifold $Z$, for all $n$ such that $|n| \leq N$.
	\item SnapPy opens, runs, and saves this file, producing a second output file named \texttt{[OUT] K [TEST].py}.
	\item The program verifies if this last file contains all $2 N + 1$ cases. This step is included because the run time in SnapPy is sometimes unpredictable.
	\item The program identifies the knots whose volume is less than or equal (up to approximations) to the volume of the knot $K$. Those cases are singled out in the output file and, if possible, identified by SnapPy.
\end{enumerate}

All files corresponding to the knots listed in Theorem~\ref{thm:low-crossing} can be found on \kh{Kyle's website}. These files prove the theorem.

\begin{proof} We produced the link $L$ aforementioned for all knots $K$ mentioned in Theorem~\ref{thm:low-crossing}. All the DT codes are included in the file \texttt{DT\char`_List.txt}, which is correctly formatted for the script \texttt{integral\char`_slopes.py}. Running it outputs the \texttt{.py} files labeled \texttt{[OUT]}, and text that is reproduced in the file \texttt{output.txt}. From reading this file, we see that:
	\begin{enumerate}[label=\normalfont (\alph*)]
		\item For all knots $K$ of crossing number at most 10 with unknotting number $u(K) = 1$ and not twist knots, the program indicates that only $n = 2$ or $n = -2$ could be characterizing.
		\kh{\item For $K$ one of the knots $8_4$, $8_6$, $8_{10}$, or $8_{12}$, the program indicates that all knots $K_{n}'$ have different volume than $K$, so no integral slopes can be characterizing.}
		\item For the knot $8_1$, the program indicates that only $n = 0$ could be characterizing.
	\end{enumerate}
\end{proof}

\subsection{Possible extensions of Theorem 1.1}

Theorem~\ref{thm:low-crossing} is stronger than Theorem~\ref{thm:unknotting-one} for low-crossing number knots, since it is valid for knots $K$ with unknotting number $u(K) > 1$, and even for some twisted Whitehead doubles. 

To use Piccirillo's construction, the original band presentation is not necessary. Any link $L' = K\cup R$, where $R$ is a one-handled unknot can receive a meridian $c$ to become the link $L$ described in section~\ref{sec:code}. This link fits Piccirillo's construction, and will yield a non-trivial twist family of knots $K_{n}'$ if and only if $R$ is not a meridian to $K$. As proved, the banded Hopf Link presentation will always yield such $L'$ through a single handle slide, but we have encountered diagrams for this link that do not depend on a band presentation.

We prove parts (b) and (c) of Theorem~\ref{thm:low-crossing} using these diagrams. We could not find an algorithm to produce them, but a few strategies were recurrently successful:

\begin{enumerate}
	\item Start with a simple banded Hopf link diagram, then change the framing on the band and perform the handle slide.
	\item Start with a simple banded Hopf link diagram for a knot $K$ and perform the usual handle slide. Then, perform a second handle slide of $K$ over $R$ on a region where $K$ is not parallel to $R$. Sometimes, different framings on the second handle slide successfully yielded different links fitting Piccirilo's construction.
	\item Start from a simple diagram $B \cup R \cup G$ that fits Piccirilo's construction. Then, repeatedly slide $B$ over $R$ until it comes off of $G$.
\end{enumerate}

Such strategies take diagrams for knots we previously find and modify them in order to find analogous for other knots. Whether the two knots related by these operations have any relationship is unknown. If real, such relationship could lead to an algorithm that produces a link $L$ for $J$ with $u(J) > 1$ given the link $L$ for $K$ with $u(K) = 1$. Moreover, using the third strategy we were even successful to find a link $L$ that fits Piccirilo's construction for the twist knot $8_1$.

We now pose a question that would allow us to generalize Theorem~\ref{thm:unknotting-one} for more knots, dropping the assumptions on the unknotting number and the twisted Whitehead classification.

\begin{question}
	For what knots $K$ is there a link $K \cup R$, where $R$ is an unknot which is not a meridian to $K$ and such that $(0,\hs0)$-surgery on $K\cup R$ yields $S^3$?
\end{question}

\section{Additional results}

\kh{[Any additional findings, including HFK or Khovanov homology findings, or enhanced sliceness obstructions using Piccirillo's technique, can go here.]}


\bibliographystyle{alpha}
\bibliography{biblio}

\end{document}
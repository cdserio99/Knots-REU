\documentclass[11pt,usenames,dvipsnames,reqno]{amsart} 
%\usepackage[margin=1.34in]{geometry}
\usepackage{xypic}
\usepackage{amsmath}
\usepackage{amssymb}
\usepackage{amsthm}
%\usepackage[notref,notcite]{showkeys}
\usepackage[colorlinks=true,linkcolor=NavyBlue,citecolor=NavyBlue,anchorcolor=NavyBlue,urlcolor=NavyBlue]{hyperref}
\usepackage{graphicx}
\usepackage{enumitem}
\usepackage{tikz}

%Shortcuts
\newcommand{\cc}{\mathbb{C}}
\newcommand{\zz}{\mathbb{Z}}
\newcommand{\rr}{\mathbb{R}}
\renewcommand{\sl}{\mathrm{sl}}
\newcommand{\tb}{\mathrm{tb}}
\newcommand{\lk}{\mathrm{lk}}
\newcommand{\st}{{\mathrm{st}}}
\newcommand{\id}{\operatorname{id}}
\newcommand{\pr}{\operatorname{pr}}
\newcommand{\Symp}{\operatorname{Symp}}
\newcommand{\Mod}{\operatorname{Mod}}
\newcommand{\Diff}{\operatorname{Diff}}
\newcommand{\km}{{\normalfont{(KM)}}}
\newcommand{\vol}{\operatorname{vol}}


%Environments
\newtheoremstyle{thm}{15 pt}{10 pt}{\itshape}{}{\bfseries}{.}{.5em}{}
\newtheorem{thm-main}{Theorem}
\newtheorem{theorem}{Theorem}
\numberwithin{theorem}{section}
\newtheorem{corollary}[theorem]{Corollary}
\newtheorem{question}[theorem]{Question}
\newtheorem{lemma}[theorem]{Lemma}
\newtheorem{conjecture}[theorem]{Conjecture}
\newtheorem{proposition}[theorem]{Proposition}
\newtheorem*{thm*}{Theorem}
\newtheorem*{cor*}{Corollary}

\newtheoremstyle{ex}{10 pt}{10 pt}{\normalfont}{}{\bfseries}{.}{.5em}{}
\theoremstyle{ex}
\newtheorem{example}[theorem]{Example}
\newtheorem{definition}[theorem]{Definition}

\newtheoremstyle{rem}{10 pt}{10 pt}{\normalfont}{}{\it}{.}{.5em}{}
\theoremstyle{rem}
\newtheorem{remark}[theorem]{Remark}

\usepackage{xcolor}
\def\kh#1{\textcolor{Blue}{#1}}
\begin{document}

%%%%%%%%%%%%%%%%%%%%%%%%
%                     End of preamble                       %
%%%%%%%%%%%%%%%%%%%%%%%%


\title{Integer characterizing slopes and unknotting numbers}

\begin{abstract} 
\kh{[Write one!]}
\end{abstract}


\maketitle

\section{Introduction}\label{sec:intro}

\kh{[Introduction with some motivation and introduction of terms like ``characterizing slopes''. See the papers by Yi and Zhang and by McCoy and by Lackenby on characterizing slopes for ideas to steal. Mention Piccirillo's results for unknotting number one and using this to show that the Conway knot isn't slice. Mention that McCoy's work \cite{mccoy:hyperbolic}, which shows that a hyperbolic knot has only finitely many non-characterizing slopes $p/q$ with $|q| \geq 3$. In a sense, this implies that ``most'' slopes $p/q$ are characterizing for any given hyperbolic knot $K$: The  probability that a randomly chosen slope $p/q$ is characterizing approaches 1 as $|p|+|q|\to \infty$.]}



\begin{theorem}\label{thm:unknotting-one} If a knot $K$ has unknotting number one and is not a twisted Whitehead double, then $K$ has at most finitely many integer characterizing slopes.
\end{theorem}

\kh{[Mention the following findings for knots with low crossing number. Also mention that it shows that having at most finitely many integer characterizing slopes is not special to knots with unknotting number one.]}


\begin{theorem}\label{thm:low-crossing} For knots $K$ with crossing number $c(K) \leq 10$:
\begin{enumerate}[label=\normalfont \bf (\alph*)]
\item If $K$ has unknotting number one and is not a twist knot, then $K$ has at most one integer characterizing slope.
\item If $K=8_4$, $8_6$, $8_{10}$, or $8_{12}$, then $K$ has $u(K)>1$ and has no integer characterizing slopes.
\end{enumerate}
\end{theorem}

\kh{[Can we push this to $c(K)\leq 12$? Can we expand on the list of knots with $u(K)>1$?]} 
\kh{[Mention that these results affirmatively answer a question of Baker and Motegi \cite[Question~1.7]{baker-motegi}, who asked if there exist knots with fewer than 8 crossings that have infinitely many noncharacterizing slopes.]}


\begin{conjecture}
Every integer surgery on a twisted Whitehead double is characterizing.
\end{conjecture}

\kh{[Remark: Together with Theorem~\ref{thm:unknotting-one}, this conjecture would imply that a knot with unknotting number one has infinitely many integer characterizing slopes if and only if it is a twisted Whitehead double. It suffices to show that if $S^3_n(K')=S^3_n(Wh(J))$, then $g(K')\ll |n|$. Note that $g(Wh(J))=1$.]}






\subsection{Non-characterizing slopes and dual knots} Let $Y$ be a 3-manifold obtained from $p/q$ surgery on a knot $K\subset S^3$. Then $Y$ is obtained by thickening $K$ to a solid torus $N(K)$, removing the interior of $N(K)$ from $S^3$, and gluing a solid torus back into $S^3\setminus\mathrm{int}(N(K))$ in a different manner. Note that $K$ is the core of the solid torus $N(K)$.

\begin{conjecture}[Baker]
\kh{[Fill this in.]}
\end{conjecture}

\subsection{Code and data} \kh{[Summarize how computer calculations factor into the proof of Theorem~\ref{thm:low-crossing} and the results on Baker's conjecture. Ultimately, we'll make the files publicly available, and this subsection can describe where to find those files.]} \kh{[Also, we can probably get Gabriel's python script to work for any link L that fits into Piccirillo's construction. That would be useful.]}



\section{Knots with unknotting number one}\label{sec:unknotting-one}

\subsection{Twist families of knots}

\kh{[Define what it means to produce a twist family of knots $K_n$ from a two-component link $K\cup C$ where $C$ is an unknot that is linked with $K$. A picture would help.]}

\kh{[Recall Theorem~1.3 from Baker-Motegi, if that's what's going to be used.]}

\kh{[Note that any knot fitting into Baker and Motegi's construction fits into Piccirillo's construction.]}

\subsection{Proof of Theorem~\ref{thm:unknotting-one}}

\section{Knots with low crossing number}

\subsection{Banded Hopf link diagrams} \kh{[Describe these and how they produce diagrams that fit into Baker and Motegi's construction. Give examples for a couple knots with unknotting number one, plus an example for a knot with unknotting number two.]}


\subsection{Hyperbolic Dehn surgery} \kh{[Recall and discuss the relevant theorems used in our approach.]}

\subsection{Proof of Theorem~\ref{thm:low-crossing}}

\kh{[Desribe the two-part algorithm we use to rule out integer characterizing slopes, beginning with finding $N \in \zz_{>0}$ such that $\vol(S^3 \setminus K'_n)> \vol(S^3 \setminus K)$ for all $|n|>N$ and then directly examining $K'_n$ for $|n| \leq N$.]}

\section{Additional results}

\kh{[Any additional findings, including HFK or Khovanov homology findings, or enhanced sliceness obstructions using Piccirillo's technique, can go here.]}


\bibliographystyle{alpha}
\bibliography{biblio}

\end{document}
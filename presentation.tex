\documentclass{beamer}
\usepackage{xypic}
\usepackage{amsmath}
\usepackage{amssymb}
\usepackage{amsthm, bbm}
\usepackage{graphicx}
\usepackage{tikz}

%Shortcuts
\newcommand{\cc}{\mathbb{C}}
\newcommand{\zz}{\mathbb{Z}}
\newcommand{\rr}{\mathbb{R}}
\renewcommand{\sl}{\text{sl}}
\newcommand{\tb}{\text{tb}}
\newcommand{\lk}{\text{lk}}
\newcommand{\st}{{\text{st}}}
\newcommand{\id}{\operatorname{id}}
\newcommand{\pr}{\operatorname{pr}}
\newcommand{\Symp}{\operatorname{Symp}}
\newcommand{\Mod}{\operatorname{Mod}}
\newcommand{\Diff}{\operatorname{Diff}}
\newcommand{\km}{{\normalfont{(KM)}}}
\newcommand{\vol}{\operatorname{vol}}
\newcommand{\defeq}{\mathrel{\vcenter{\baselineskip0.5ex\lineskiplimit0pt\hbox{\scriptsize.}\hbox{\scriptsize.}}}=}
\newcommand{\hs}{\hspace{0.05em}} % half-space


%Environments
\newtheoremstyle{thm}{15 pt}{10 pt}{\itshape}{}{\bfseries}{.}{.5em}{}
\newtheorem{thm-main}{Theorem}
\newtheorem{conjecture}[theorem]{Conjecture}
\newtheorem{proposition}[theorem]{Proposition}
\newtheorem*{thm*}{Theorem}
\newtheorem*{cor*}{Corollary}

\newtheoremstyle{ex}{10 pt}{10 pt}{\normalfont}{}{\bfseries}{.}{.5em}{}
\theoremstyle{ex}

\newtheoremstyle{rem}{10 pt}{10 pt}{\normalfont}{}{\it}{.}{.5em}{}
\theoremstyle{rem}
\newtheorem{remark}[theorem]{Remark}


\title[Integer Characterizing Slopes]{Knot Surgery and Integer Characterizing Slopes}
\usetheme{Madrid}

\author[Agostini, Chen, Serio, Wang, Wu, and Wu\;\;\;]{Gabriel Agostini, Sophia Chen, Christian Serio, Cecilia Wang, Anton Wu, and Kexin Wu\\
Advisors: Kyle Hayden and Aliakbar Daemi}
\institute{Columbia University}
\date{August 1, 2019}

\begin{document}
	
	\begin{frame}
		\titlepage
	\end{frame}

	\begin{frame}
		\frametitle{Knots and links in the 3-sphere}
		\begin{definition}
			A \textit{knot} $K$ is the image of a smooth embedding of the circle $S^1$ into a 3-manifold, usually the $3$-sphere $S^3$. In particular, $K$ is diffeomorphic to $S^1$. A \textit{link} $L$ is a disjoint union of knots, which may be knotted together.
		\end{definition}
		\begin{definition}
		Let $M,N$ be manifolds and $g,h\!:N\to M$ embeddings. An \textit{ambient isotopy} of $M$ carrying $g$ to $h$ is a continuous map $F\!:M\times[0,1]\to M$, such that $F_t=F(\cdot,t)$ is a homeomorphism of $M$ for each $t\in[0,1]$,  $F_0=\mathbbm{1}$, and $F_1\circ g = h$.
		\end{definition}
		\begin{itemize}
		\item We regard two knots $K,K'\subset S^3$ to be equivalent if they differ by an ambient isotopy of $S^3$. We write $K\simeq K'$.
		
		\item Equivalently, we can view knots as subsets of $\mathbb{R}^3$ rather than $S^3$.
		
		\end{itemize}
	\end{frame}

	\begin{frame}
		\frametitle{Knot diagrams}
		\begin{itemize}
			\item We can study a knot $K\subset\mathbb{R}^3$ by projecting it onto a hyperplane $\mathbb{R}^2$.
			
			\item If $\pi\!:\mathbb{R}^3\to\mathbb{R}^2$ is a projection such that $\pi(K)$ is an embedded curve except at finitely many \textit{crossing points}, then $\pi(K)$ is a \textit{diagram} for $K$.
			
			\item The \textit{crossing number} $c(K)$ is the minimum number of crossings in a diagram of $K$.
	
		\end{itemize}
	
		\begin{columns}[onlytextwidth]
			\begin{column}{.4\textwidth}
				\begin{figure}
					\includegraphics[width=\textwidth]{examples}
					\caption{Knots with $c(K)\leq 7$}
				\end{figure}
			\end{column}
			\hfill
			\begin{column}{.4\textwidth}
				\begin{figure}
					\includegraphics[width=\textwidth]{hopf}
					\caption{Hopf link}
				\end{figure}
			\end{column}
		\end{columns}
	\end{frame}

	\begin{frame}
		\frametitle{Reidemeister moves}
		
		\begin{theorem}[Reidemeister]
			Two knots $K,K'$ are isotopic if and only if they have diagrams that differ by a sequence of planar isotopies and Reidemeister moves.
		\end{theorem}
		
		\begin{center}
			\includegraphics[scale=0.7]{reidemeister.jpg}
		\end{center}
	\end{frame}

	\begin{frame}
		\frametitle{Crossings}
		\begin{itemize}
			\item If we orient a knot $K$, then we can define a \textit{sign} for each crossing by the right-hand rule.
			
			\item For a two-component link $L=K\cup K'$, the \textit{linking number} $\lk(L)$ is one half the sum of the signs of the crossings between $K$ and $K'$ in a diagram of $L$.
		\end{itemize}
	
		\begin{center}
			\includegraphics[scale=0.25]{signs}
		\end{center}
	\end{frame}

	\begin{frame}
	\frametitle{Piccirilo's construction: Knots with the same surgery}
	\begin{theorem}[Piccirilo 2018]
		Let $L = R \cup G \cup B$ be a surgery diagram for some 3-manifold $Y$ such that:
		\begin{enumerate}
			\item $R$ is a zero-framed unknot, $B$ and $G$ have integral framings.
			\item Ignoring $B$, $R$ is isotopic to a meridian of $G$.
			\item Ignoring $G$, $R$ is isotopic to a meridian of $B$.
			\item $B$ and $G$ have linking number 0.
		\end{enumerate}
		Then, there exist knots $K$ and $K'$ such that $Y = S^3_{n}(K) = S^3_{n}(K')$.
	\end{theorem}
	\begin{itemize}
		\item Piccirilo's construction comes from an older construction, the \textit{dualizable patterns} construction, to produce knots with the same surgery.
	\end{itemize}
\end{frame}

\begin{frame}
	\frametitle{Piccirilo's construction (cont.)}
	\begin{figure}
		\begin{center}
			\includegraphics[width=3in]{picci.png}
			\caption{Diagram of a link $L$ used by Piccirilo on her original paper. The colors (red, blue, and green) match the component letters $R$, $B$, $G$.}
		\end{center}
	\end{figure}
\end{frame}


	\begin{frame}
		\frametitle{Baker-Motegi: Knots with the same surgery}
			\begin{itemize}
				\item We adapted the following construction from Baker and Motegi (2018).
				
				\item Let $K$ be a knot, and suppose we can take an unknot $c$ linked with $K$ such that $(0,0)$-surgery on $K\cup c$ is $S^3$.
				
				\item Baker-Motegi present a method for producing knots $K_n'$ with $S^3_n(K)\cong S^3_n(K_n')$ from this link.
				
				\item Define $K'$ to be the surgery dual to $c$ in $S^3=S^3_{(0,0)}(K\cup c)$.
				
				\item Then $K'$ has the same 0-surgery as $K$.
				
			\end{itemize}
		
			
	
	\end{frame}

	\begin{frame}
		\frametitle{Baker-Motegi (cont.)}
		\begin{itemize}
			
			\item Also define $c'$ to be the surgery dual to $K$ in $S^3$.
			
			\item After some Kirby calculus, we find that in the surgered manifold $S^3=S^3_{(0,0)}(K\cup c)$, $c'$ is an unknot linked with $K'$.
			
			\item Let $K_n'$ be the result of twisting $K'$ through $c'$, $n$ times.
			
			\item Then $K_n'$ has the same $n$-surgery as $K$ for all $n$.
			
			\item Moreover, if $c$ is not a meridian to $K$, then $K\simeq K_n'$ for at most finitely many $n$.
		\end{itemize}
	\end{frame}

	\begin{frame}
		\frametitle{Baker-Motegi Illustration}
		\begin{center}
			\includegraphics[scale=0.5]{bm}
		\end{center}
	\end{frame}

	\begin{frame}
		\frametitle{Obtaining a link $K\cup c$ when $u(K)=1$ (Piccirillo)}
		\begin{itemize}
			\item In the special case where $K$ has unknotting number one, we can use a band presentation for $K$.
			
			\item We start with a Hopf link $R\cup B$ and slide one component over the other according to the band presentation for $K$.
			
			\item Then $R$ remains an unknot, which we rename $c$, and $B$ becomes the knot $K$.
			
			\item After adding a meridian to $R$ and sliding $B$ over $R$, we get a diagram that also fits Piccirillo.
			
			\item Example: $7_7$
			
			\begin{center}
				\includegraphics[scale=0.4]{k_cup_c_example1}
				\includegraphics[scale=0.4]{k_cup_c_example2}
			\end{center}
		\end{itemize}
	\end{frame}

	\begin{frame}
		\frametitle{$K\cup c$ for $u(K)=1$ (cont.)}
		\begin{lemma}
			Let $K\cup c$ be the link obtained by the handle slide above. Then $(0,0)$-surgery on $K\cup c$ gives $S^3$.
		\end{lemma}
		\begin{proof}
		After the handle slide, the framing of $c$ remains 0, and the framing of $K$ changes by $\pm 2$ depending on the linking number of the Hopf link. If we adjust the framing of the blue component of the Hopf link to $\mp 2$ before the slide, then we see that $(0,0)$-surgery on $K\cup c$ is the same as $(\mp 2,0)$ surgery on a Hopf link. It can be shown that $(n,0)$-surgery on a Hopf link is $S^3$ for any $n\in\mathbb{Z}$.
		\end{proof}
	\end{frame}

	\begin{frame}
		\frametitle{Obtaining $K\cup c$ when $u(K)>1$}
		
		\begin{itemize}
			\item We do not have a systematic way of finding a link $K\cup c$ for a given knot $K$ with $u(K)>1$.
			
			\item If we perform multiple slides on a Hopf link, then we obtain a link $K\cup c$ for some knot $K$ with higher unknotting number, but we have no way of predicting what $K$ will be.
			
			\item Example: $10_{125}$
			
			\begin{center}
				\includegraphics[scale=0.25]{10_125_1}
				\includegraphics[scale=0.25]{10_125_2}
				\includegraphics[scale=0.25]{10_125_3}
			\end{center}
		\end{itemize}
	\end{frame}

\begin{frame}
	\frametitle{Applications of the construction}
	\begin{itemize}
		\item We have produced a banded Hopf link presentation for all the 78 unknotting number one knots $K$ with $c(K)\leq 10$.
		\item We followed the above procedure to manually produce a diagram for a knot $K'$ that has the same zero-surgery as $K$, using a software called KLO which can perform Kirby calculus on link diagrams.
		\item Using a software called SnapPy, we were able to verify that whenever $K$ was not a twist knot, $K'$ was not isotopic to $K$. Thus, zero was not a characterizing slope.
	\end{itemize}
	\medskip
	\begin{itemize}
		\item \textbf{Question:} Can we classify all integer slopes once we have the link $K\cup c$?
	\end{itemize}
\end{frame}


	\begin{frame}
	\frametitle{Ruling Out Integer Characterizing Slopes}
	\begin{itemize}
		\item We developed an algorithm that can be carried out by a computer script to rule out the integer characterizing slopes of some knot $K$, once we have manually produced the link $L = K \cup c$.
	\end{itemize}
	\begin{enumerate}
		\item Given a link $L$, run SnapPy commands to find out the volume of the knot $K$, the volume of the manifold $Z = S_0^3(c)$, and the length of the Seifert longitude in $Z$.
		\item Using these values, apply the theorems of hyperbolic Dehn surgery to find the bound $N$ such that any integer $|n| > N$ is a characterizing slope for $K$.
		\item  For the remaining $2N + 1$ cases, verify if the volume of the knot with the same $n$-surgery as $K$ matches the volume of $K$.
	\end{enumerate}
	\begin{itemize}
		\item We use the \textit{DT code} of the link, which uniquely describes all links that we deal with, up to isotopy.
	\end{itemize}
\end{frame}

\begin{frame}
\frametitle{Findings}
\begin{theorem}[Low-Crossing Knots]
	Regarding the integer slopes of knots $K$ such that $c(K) \leq 10$:
	\begin{itemize}
		\item If $K$ has unknotting number $u(K) = 1$ and $K$ is not a twist knot, then $K$ has at most \textit{one} integer characterizing slope, namely $\pm 2$.
		\item If $K$ is the twist knot $8_1$, then $K$ has at most \textit{one} integer characterizing slope, namely $0$.
		\item If $K$ is one of the $u(K) = 2$ knots $8_4$, $8_6$, $8_{10}$, $8_{12}$, $8_{16}$, $10_{148}$, $10_{149}$, or $10_{150}$, $K$ has no possible integer characterizing slope.
		\item If $K$ is one of the $u(K) = 2$ knots $8_3$, $10_{125}$, or $10_{126}$, $K$ has at most \textit{one} integer characterizing slope.
	\end{itemize}
\end{theorem}
\end{frame}




	\begin{frame}
		\frametitle{Proof of Theorem 1 (cont.)}
		\begin{itemize}
			\item Recall that  for $K$ with $u(K)=1$, we could obtain a link $K\cup c$ as in Baker-Motegi with $(0,0)$-surgery $S^3$ by sliding over a Hopf link according to the band presentation for $K$.
			
			\item It remains to show that if $K$ is not a twisted Whitehead double, then $c$ is not a meridian to $K$ after the handle slide.
			
			\item In this case, in any band presentation for $K$, the band must cross the disc bounded by one of the components of the Hopf link.
			
		\end{itemize}
	
		\begin{lemma}
			Let $R\cup B$ be a Hopf link, and consider a handle slide of $R$ over $B$ yielding a link $L$ in which $R$ remains a meridian to $B$. Then there exists a handle slide of $R$ over $B$, yielding a link isotopic to $L$, along a band that does not cross either of the discs bounded by $R$ or $B$.
		\end{lemma}
	\end{frame}

\begin{frame}
	\frametitle{Extension of Theorem 1}
	\begin{itemize}
		\item How can we drop the condition on Twisted Whitehead Doubles? And on unknotting number?
		\item Can we produce an algorithm to find links $L = K \cup c$ from handle slides on a Hopf link?
	\end{itemize}
	\begin{conjecture}
		If $K$ is a knot with unknotting number $u(K) = 1$ and $K$ is not a twist knot, then $K$ has at most \textit{one} integer characterizing slope: $\pm 2$.
	\end{conjecture}
\end{frame}

\begin{frame}
	\frametitle{Next Steps}
	\begin{itemize}
		\item Rigorously prove the lemma on band presentations.
		\item Keep experimenting with handle slides in order to expand the list on Theorem 1.2.
		\item Attempt to use other tools to prove a version of Theorem 1.1 for Twisted Whitehead Doubles.
		\item Look into our final conjecture.
	\end{itemize}
\end{frame}

\end{document}

